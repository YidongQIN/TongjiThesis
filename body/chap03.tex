\documentclass[../Main/thesis]{subfiles}
\begin{document}
\chapter{新增内容测试}
\label{cha:new}

之前的两章都是原同济LaTeX模板中的内容。
这一章开始,就是修改、或新加内容的测试。
\cite{RN2018}

当然,不变的是满纸荒唐言,用于填充内容。
\cite{RN2361}

\section{下划线测试}
\label{sec:underline}

测试下划线。虽然是不常用的功能,决定是否留在 cls 文件中。

这是默认的\underline{下划线功能}的效果。

% 这是CJK字体的\CJKunderline[*]{特殊下划线}宏包的效果。

\section{多级标题} \label{sec:seclv}

原本的 \textsf{romantitle} 命令没有用。
验证发现,是因为 \textsf{CTeX} 宏包中的字体命令(\textsf{$\backslash$songti}、\textsf{$\backslash$heiti}、\textsf{$\backslash$fangsong}、\textsf{$\backslash$lishu})等,只对汉字字体生效。
究其原因,因为是 \textsf{xeCJK} 的设计。

于是,重写了控制章节标题的数字编号的命令。

\subsection{二级节标题} \label{ssc:subsec2}

二级节标题形如“1.3.1 xxx”。
格式为:黑体,四号,顶左,单倍行距,段前 12 磅,段后 6 磅,序号与题名间空一个字符。

\subsubsection{三级节标题} \label{sss:subsub3}

三级节标题形如“1.3.1.1 xxx”。
格式为:黑体,小四顶左,单倍行距,段前空 12 磅,段后空 6 磅,序号,且与题名间空一个字符。

\paragraph{四级标题即段落} \label{par:par4}

这是本次新增的层次命令。

四级标题形如:“(1) xxx”。
格式是仿照正文的,区别就在黑体字体。

\paragraph{另一个四级标题} \label{spr:subpar}

至此,已经有了从0级chapter到4级paragraph共5个级别了,应该够了吧。

\paragraph{如果必须要有更低的层次}

就像必须要有巫妖王一样。

\subsubsection{段落编号}

需要更加完善地测试新命令。
现在测试用 subsubsection 分割。

\paragraph{一个新段落}

多补充几个段落测试。
经过 subsubsection 的分隔,现在应该是全新的段落了。

\paragraph{两个段落} \label{par:second}

正文格式是宋体,小四号(英文用 Times New Roman 体,小四号),两端对齐书写,段落首行左缩进 2 个汉字符。行距 20 磅(段落中有数学表达式时,可根据表达需要设置该段的行距),段前段后 0 磅。

\subsection{最后的段落测试} \label{sec:lastpar}

如果“跨级”中断呢?
比如这个,用section分隔之后,直接用paragraph会不会重新编号呢。

一下就用伟大的 Git 教程测试一下吧。

\paragraph{基本概念} \label{par:git-concept}

\paragraph{工作区}
working directory = workspace,就是电脑中的目录。

\paragraph{暂存区}

stage = staging area=index
而repository中最重要的就是称为stage(或者叫index)的暂存区
每次修改,如果不用git add到暂存区,那就不会加入到commit中

\paragraph{版本库}

repository = git directory =(local +remote)
可以简单理解成目录下隐藏的.git文件夹。

\paragraph{远程仓库}
remote = remote repository
一个主机,作为团队公用的仓库。一般就是GitHub、bitbucket、gitee等git服务商。
本质上是一个git仓库的地址,形如 git@xxx.com:username/project。
在远程仓库管理中,会给它一个别名,通常是origin表示。

\paragraph{分支}
branch
一个仓库会有多个分支,对用不同的开发进度。
约定俗成的主分支叫master,也就是对外展示用的发布版本。

\subsection{基本流程} \label{ssc:git-workflow}

其实应该顺便测试列表环境的,这里就偷懒吧。
调整本地工作目录
2. 创建项目
a. 在git网站建立好新项目,把库克隆到本地
git clone http://
b. 或者,直接在本地文件夹目录下初始化
git init
(本地初始化后如要连接网络远程仓库可以再用 git remote add)
3. 工作区 -> 暂存区:add添加(也叫stage)
a. 保存修改并添加文件到暂存区(index,或叫staging area)
git add <文件名>
常用简化命令如下
git add -A % stages all changes
git add .  % stages new files and modifications, without deletions
git add -u % stages modifications and deletions, without new files
b. 或者查看区别
git diff
4. 暂存区 -> 版本库:commit提交
git commit -a -m “message”
-a表示全部 -m 添加一个留言
○ 如果弹出了vim截面,则用i命令进入编辑模式,输入完成后,esc退回到命令模式,
○ 然后:wq保存并退出vim,即可回到git命令窗口。
5. * 确认提交状况  git status
6. 推送 git push -branch
之前步骤1-5都是在本地的版本管理,直到这一步才牵扯到远程仓库。

\section{列表环境测试} \label{sec:list}

列表也是很常用的排版方式。
最方便的方法就是调用 enumerate 宏包定制标签,用 A a R r 1 分别代表大写字母、小写字母、大写罗马、小写罗马和阿拉伯数字编号。

为了偷懒和整齐,这次对一到三级列表的默认样式进行了设计。

\subsection{基本列表} \label{ssc:basiclist}

首先,先熟悉一下基本列表环境。
最常用的是有序列表。

\begin{enumerate}
  \item 中文 gb7714-2015.bbx
  \item gb7714-2015.bbx
  \item 中文 gb7714-2015ay.bbx
  \item gb7714-2015ay.bbx
\end{enumerate}

也挺常用的无序列表。

\begin{itemize}
  \item 中文 gb7714-2015.bbx
  \item gb7714-2015.bbx
  \item 中文 gb7714-2015ay.bbx
  \item gb7714-2015ay.bbx
\end{itemize}

本次修改的发起者,用description列表。
感觉这是一种不怎么常用的列表:大概是因为 word 中没有吧哈哈哈。

\begin{description}
  \item[CNchence] 在github.com CNchence
  \item[marquistj13] 在github.com marquistj13
  \item[Wenda] 在github.com Williamwenda
  \item[CNchence] 在github.com CNchence
  \item[Wenda] 在github.com Williamwenda
\end{description}

更常见的是嵌套列表。

\subsection{嵌套}

列表的内容就很随意了。
查看commit的历史it log pretty=oneline 会显示历次commit的一个SHA1码,以及现在HEAD指向的版本。
上一个版本就是HEAD,上上一个版本就是HEAD,往上100个版本成HEAD~100。
回滚版本 git reset --hard HEAD
有时候不识别,会追问 More? 那么就用HEAD~1指向上一个版本。
回滚“回滚”的版本 reset到旧版本之后,新版本不会显示在log中。但是仍然可以用那个十六进制的版本号找回。
git reset --hard nnnn版本号前几位,版本号没必要写全,前几位就可以了,Git会自动去找。 git reflog 记录你的每一次命令
对比功能 git diff \#其参数解释如下图。

\begin{enumerate}
  \item 介绍了时间序列分析的相关概念与理论。
  \item 重点对时间序列的平稳性特点、检验方法及平稳模型体系做了详细阐述。
  \begin{enumerate}
    \item 为充分反映出监测序列的季节周期性,建立了一套基于单测点监测序列的季节 ARIMA 模型分析方法。
    \item 对建模过程中的平稳性检验、模型建立步骤、定阶、参数估计以及检验进行了系统的阐述,
  \end{enumerate}
  \item 并对玉峰大桥某监测测点的实测数据进行了季节 ARIMA 建模与预测分析。
  \item 结果表明季节 ARIMA 模型能较好地模拟监测序列的变化趋势,预测精度较高。
  \begin{enumerate}
    \item 用力地嵌套列表。
    \begin{enumerate}
      \item 针对监测序列中某些测点间具有较高的相似性,
      \item 探讨了滞后协整分析的参数估计、模型检验等方法,
    \end{enumerate}
    \item 最后建立了支座位移与温度间的 ADL 与 ECM 模型,
    \item 结果表明该模型具有良好的拟合精度与预测效果。
  \end{enumerate}
\end{enumerate}

建立起基于多测点序列的模型分析方法。
并排图形。已经不再推荐,用新的需要注意的是,由于我们的新模板中使用了的浮动体的配置,并且加载了,就不能再加载了,因此我们再也无法使用提供的命令了。不用这么悲伤,我们可以使用的直接将其替换即可。同时注意标题在中是必选参数,即用增加伪斜体伪粗体选项。段前距离修改关键词加粗。

主要参考资料:
\begin{itemize}
  \item 介绍了时间序列分析的相关概念与理论。
  \item 重点对时间序列的平稳性特点、检验方法及平稳模型体系做了详细阐述。
  \begin{itemize}
    \item 为充分反映出监测序列的季节周期性,建立了一套基于单测点监测序列的季节 ARIMA 模型分析方法。
    \item 对建模过程中的平稳性检验、模型建立步骤、定阶、参数估计以及检验进行了系统的阐述,
  \end{itemize}
  \item 并对玉峰大桥某监测测点的实测数据进行了季节 ARIMA 建模与预测分析。
  \item 结果表明季节 ARIMA 模型能较好地模拟监测序列的变化趋势,预测精度较高。
  \begin{itemize}
    \item 用力地嵌套列表。
    \begin{itemize}
      \item 针对监测序列中某些测点间具有较高的相似性,
      \item 探讨了滞后协整分析的参数估计、模型检验等方法,
    \end{itemize}
    \item 最后建立了支座位移与温度间的 ADL 与 ECM 模型,
    \item 结果表明该模型具有良好的拟合精度与预测效果。
  \end{itemize}
\end{itemize}


效果预览:请到这里下载PDF,在github.com marquistj13 TongjiThesis releases latest。
注意:由于我们不经常打tag,因此release界面往往不是最新的,请在本页面下载最新模板。


\begin{itemize}
  \item 介绍了时间序列分析的相关概念与理论。重点对时间序列的平稳性特点、检验方法及平稳模型体系做了详细阐述。
  \item 为充分反映出监测序列的季节周期性,建立了一套基于单测点监测序列的季节 ARIMA 模型分析方法。对建模过程中的平稳性检验、模型建立步骤、
  定阶、参数估计以及检验进行了系统的阐述,并对玉峰大桥某监测测点的实测数据进行了季节 ARIMA 建模与预测分析。结果表明季节 ARIMA 模型能较好地模
  拟监测序列的变化趋势,预测精度较高。
  \item 针对监测序列中某些测点间具有较高的相似性,建立起基于多测点序列的 ADL 模型分析方法。将滞后分析、时序分析以及滞后协整分析有机整合在一起,探讨了滞后协整分析的参数估计、模型检验等方法,并推导了完整的 ADL-ECM 误差修正模型表达方程。最后建立了支座位移与温度间的 ADL 与 ECM 模型,结果表明该模型具有良好的拟合精度与预测效果。
  \item 以时间序列的预测模型作为结构的退化模型,对各测点进行退化趋势模拟与退化临界时刻预测,从整体上把握桥梁的整体性能退化趋势与使用极限寿命。尽管目前时间序列分析方法在桥梁监测上的应用还较少见。
\end{itemize}

RequirePackage部分主要参考自thuthesis。xparse 提供了LATEX2ε的newcommand macro的一种高性能的替代,即 NewDocumentCommand。
去掉了tj的RequirePackage{hypernat},hypernat比较老了(2001年的),它的功能是使得natbib 的multiple sequential citations (e.g [3,2,1]) will be compressed to [1-3], where the '1' and the '3' are (color)linked to the bibliography.但现在不用这个hypernat也可以实现了,我推测是其他宏包自带了这些功能,不细细追究了。

下面是一个无序列表。

\begin{description}
  \item [linxdcn] (在github.com linxdcn TongjiThesis)同学汇总的wildwolf、
  \item [svandex] 在github.com svandex masthesis)、
  \begin{description}
    \item [zhao-chen] 在github.com zhao-chen TongjiThesis)的版本。
    \item [收藏有本硕博模板的zhouyuan版] 在github.com zhouyuan/tongjithesis)。
  \end{description}
\end{description}

\section{表格测试} \label{sec:tabletest}

常用的表格是三线表。

\begin{table}[htb]
\centering
\caption{这不是个图}
\label{tab:fig}
\begin{tabular}[c]{lll}
  \toprule[1.5pt]
  0 & 1 & 2 \\ % title row
  \midrule[1pt]
  0 & 1 & 2 \\ % title row
  0 & 1 & 2 \\ % title row
  \bottomrule[1.5pt]
\end{tabular}
\end{table}

\begin{table}[htb]
\centering
\caption{tabularx}
\label{tab:tabularx}
\begin{tabularx}{0.7\textwidth}{>{\raggedleft\arraybackslash}X>{\raggedleft\arraybackslash}X>{\centering\arraybackslash}X}
  \toprule[1.5pt]
  0 & 1 & 2 \\ % title row
  \midrule[1pt]
  0 & 1 & 2 \\ % title row
  0 & 1 & 2 \\ % title row
  \bottomrule[1.5pt]
\end{tabularx}
\end{table}
\section{代码环境} \label{sec:code}
常用的代码展示,基于 Listings 宏包。
\begin{lstlisting}[style=monocolor,
  caption={first lst env},
  label={no label},
  name=测试第一个代码,
  language=PythonPlus]
  # import lxml
  import requests
  from bs4 import BeautifulSoup
  import re
\end{lstlisting}

\begin{lstlisting}[language=ParamML,
  % caption=,
  % label=lst:,
  style=OpenBrIM_style,
  ]
<O  N="C4000Psi"  T="Material"  D="Concrete">
  <P N="d"  V="0.0000002248"  D="Density"/>
  <P N="E"  V="3604.9965"  D="Modulus of Elasticity"/>
  <P N="Fc28"  V="4"  D="Concrete Compressive Strength"/>
</O>
<!-- comment -->
\end{lstlisting}

由于本模板参考文献部分依赖biblatex-gb7714-2015宏包的2018版本,因此必须在安装好Texlive后更新此宏包,Texlive2018 自带的biblatex-gb7714-2015宏包依然较老,无法正常运行本模板。

\begin{lstlisting}[language=Python,
  caption=Second CODE Block,
  label=lst:2nd,
  style=colored]
  import numpy as np
  def incmatrix(genl1,genl2):
      m = len(genl1)
      #compute the bitwise xor matrix
      M1 = bitxormatrix(genl1)
      for i in range(m-1):
          for j in range(i+1, m):
              [r,c] = np.where(M2 == M1[i,j])
              for k in range(len(r)):
                  VT[(i)*n + r[k]] = 1;

                  if M is None:
                      M = np.copy(VT)
                  else:
                      M = np.concatenate((M, VT), 1)
      return M
\end{lstlisting}

不同于tabled 表~\ref{tab:fig},源代码的展示用\ref{lst:inp}不会显示开头。
好吧,是我错了。都是需要手动指定其是“图”、“表”、“代码”或“章节”。\textbackslash ref 命令只显示编号。
仔细一想,这样也好,只做最小程度的工作,格式完全可以 \textbackslash newcommand 或者用 VSCode 脚本辅助。

第~\ref{sec:code} 小节的内容也被引用了。

\begin{lstlisting}[language=Python,
  caption=中文标题没有括号,
  label=lst:outer,
  style=colorEX]
  import numpy as np
  def incmatrix(genl1,genl2):
      m = len(genl1)
      #compute the bitwise xor matrix
      M1 = bitxormatrix(genl1)
      for i in range(m-1):
          for j in range(i+1, m):
              [r,c] = np.where(M2 == M1[i,j])
              for k in range(len(r)):
                  VT[(i)*n + r[k]] = 1;
      return M
\end{lstlisting}

推荐使用其包管理器更新此宏包,如若更新失败(我更新了好几次都失败了,它还把我的这个宏包给删了,郁闷),那么可按照官方页面在github.com/hushidong/biblatex-gb7714-2015的安装介绍,在你的tex安装目录搜索biblatex-gb7714-2015文件夹,将该文件夹中的四个文件替换为官方页面在github.com/hushidong/biblatex-gb7714-2015提供的文件即可。

这里测试三个代码片段的引用。

代码片段,没有前缀,直接中括号,见 \ref{no label} 代码。

代码片段,是按照现有的snippet直接插入的样式 \ref{lst:2nd} 代码。

代码片段, \ref{lst:outer} 代码,手动在环境之后加了label,可以被 vscode 插件检测到。

代码片段, \ref{lst:parenthesis} 代码,仅供测试,caption 加了花括号。

貌似不能从ref命令中自动调用。难道需要手动加 label 更好吗?
\begin{lstlisting}[language=Python,
  caption={Must have Parenthesis},
  label=lst:parenthesis]
  import numpy as np
  def incmatrix(genl1,genl2):
      m = len(genl1)
      #compute the bitwise xor matrix
      M1 = bitxormatrix(genl1)
      for i in range(m-1):
          for j in range(i+1, m):
              [r,c] = np.where(M2 == M1[i,j])
              for k in range(len(r)):
                  VT[(i)*n + r[k]] = 1;
      return M
\end{lstlisting}

还可以直接导入。
\lstinputlisting[language=Python,
  caption=inputted Python,
  label=lst:inp,
  style=monocolor,
  ]{../code/tmp.py}


\section{行内原样抄写并带有双引号命令}
\label{sec:newfunction}

考虑到会出现一些后缀名或者文件名之类的内容,所以新添了一个命令:
\verb|\tvt|,
其作用是在一个原样抄写的文字外围加上双引号。

CIS/2标准使用\tvt{*.stp}中性格式文件保存数据模型的数据。
对应上述梁的大纲,该梁在\texttt{*.stp}中性格式文件中的主要内容如下.


\end{document}
