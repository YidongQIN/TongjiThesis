% !TeX root = ../Main/thesis.tex
% !TEX program = xelatex
\documentclass[../Main/thesis]{subfiles}
\begin{document}
\chapter{新增内容测试}
\label{cha:new}

之前的两章都是原同济LaTeX模板中的内容。
这一章开始,就是修改、或新加内容的测试\cite{mohrssBIM}。

当然,不变的是满纸荒唐言,用于填充内容\cite{HowDigital}。

只不过偷了个懒:用\textsf{zhlipsum} 宏包填充废话充篇幅。

\section{多级标题} \label{sec:seclv}

先填充一些废话。\zhlipsum[4]

\subsection{多级标题样式修正}
\label{ssc:titles-style}

问题描述:原本的 \textsf{romantitle} 命令没有用。

验证发现,是因为 \textsf{CTeX} 宏包中的字体命令(\textsf{$\backslash$songti}、\textsf{$\backslash$heiti}、\textsf{$\backslash$fangsong}、\textsf{$\backslash$lishu})等,只对汉字字体生效。
究其原因,因为是 \textsf{xeCJK} 宏包的设计,使得CJK的字体可以独立于英语字体设置。

于是,重写了控制章节标题的数字编号的命令。
具体修改见 \tvt{README.md} 文件中的章节。
现在,可以在 \textbackslash documentclass[titlenum=XXX] 中启用不同的编号字体。
其中XXX可以选择如表~\ref{tab:title-number-format}所示。

\begin{table}[htbp]
\centering
\caption{\tongjithesis 章节编号样式选项}
\label{tab:title-number-format}
\begin{tabular}[c]{lll}
  \toprule[1.5pt]
  选项 & 编号字体 & 备注 \\ % title row
  \midrule[1pt]
  \texttt{rmtitlenum} & 衬线(Times New Roman) & 本模板默认选项 \\
  \texttt{sftitlenum} & 无衬线(Arial) & 原模版默认选项 \\
  \texttt{heititlenum} & 黑体 & 严格按照校方模板 \\
  \bottomrule[1.5pt]
\end{tabular}
\end{table}

\subsection{二级节标题} \label{ssc:subsec2}

二级节标题形如\tvt{1.3.1 xxx}。

格式为:黑体,四号,顶左,单倍行距,段前 12 磅,段后 6 磅,序号与题名间空一个字符。

\subsubsection{三级节标题} \label{sss:subsub3}

三级节标题形如\tvt{1.3.1.1 xxx}。

格式为:黑体,小四,顶左,单倍行距,段前空 12 磅,段后空 6 磅,序号,且与题名间空一个字符。

\paragraph{四级标题即段落} \label{par:par4}

段落 paragraph 是本次新增的层次命令。

四级标题形如\tvt{(1)xxx}。

格式为:黑体,小四号,顶左,单倍行距,段前 6 磅,段后 0 磅,序号与题名间没有空。

格式是仿照正文的,区别就在黑体字体。

\paragraph{另一个四级标题} \label{spr:subpar}

其实,段落的样式已经与序号列表有一些相似了,相差可能也就是“黑体”以及“顶格书写”了吧。

至此,已经有了从0级chapter到4级paragraph共5个级别了,应该够了吧。

\subparagraph{如果必须要有更低的层次。}
如果必须要有最低等级的标题,就像必须要有巫妖王一样。
\sout{其实就是为了满足强迫症。}
故还是把最低层级的 subparagraph 样式给定义了。

\subparagraph{五级标题即子段落。}
五级标题形如\tvt{(一)xxx}。

格式为:楷书,小四号,左侧空两格,单倍行距,段前 6 磅,段后 0 磅,序号与题名间没有空。

使用了 \textbackslash chinese 命令产生汉字数字。
并更改 \textbackslash setcounter\{secnumdepth\} 为 5。

\subparagraph{但是样式过多也不好。}
应该不会用到这么多层级吧?
一个比较明显的问题就是 \textsf{paragraph} 和 \textsf{subparagraph} 的样式,难以与各类列表高效区分。
\footnote{所以如果内容不多,推荐采用列表,见第~\ref{sec:list}节的展示。}

目前的方案,其实就是避免与正文(列表)采用同样的字体。
四级标题的黑体是继承自校方模板的前几级标题设置,而五级标题的楷体参考了国家公文写作的有关规范\cite{GBT9704}。

\subsubsection{段落编号}

需要更加完善地测试新命令。
现在测试用 subsubsection 分割后,能否正常编号。

\paragraph{一个新段落}

多补充几个段落测试。
经过 subsubsection 的分隔,现在应该是全新的段落了。

\paragraph{两个段落} \label{par:second}

正文格式是宋体,小四号(英文用 Times New Roman 体,小四号),两端对齐书写,段落首行左缩进 2 个汉字符。
行距 20 磅(段落中有数学表达式时,可根据表达需要设置该段的行距),段前段后 0 磅。

\subsection{最后的段落测试} \label{ssc:lastpar}

如果“跨级”中断呢?
比如这个,用section分隔之后,直接用paragraph会不会重新编号呢。

一下就用伟大的 Git 教程测试一下吧。

\paragraph{基本概念} \label{par:git-concept}

\paragraph{工作区}
working directory = workspace,就是电脑中的目录。

\paragraph{暂存区}
stage = staging area=index
而repository中最重要的就是称为stage(或者叫index)的暂存区
每次修改,如果不用git add到暂存区,那就不会加入到commit中

\paragraph{版本库}
repository = git directory =(local +remote)
可以简单理解成目录下隐藏的.git文件夹。

\paragraph{远程仓库}
remote = remote repository
一个主机,作为团队公用的仓库。一般就是GitHub、bitbucket、gitee等git服务商。
本质上是一个git仓库的地址,形如 git@xxx.com:username/project。
在远程仓库管理中,会给它一个别名,通常是origin表示。

\paragraph{分支}
branch
一个仓库会有多个分支,对用不同的开发进度。
约定俗成的主分支叫master,也就是对外展示用的发布版本。

\subsection{Git工作基本流程} \label{ssc:git-workflow}

顺便测试一下多级列表。

\begin{enumerate}[1.]
  \item 调整本地工作目录
  \item 创建项目
  \begin{enumerate}[a.]
    \item 在git网站建立好新项目,把库克隆到本地 \\

    \verb| git clone http:// |

    \item 或者,直接在本地文件夹目录下初始化 \\

    \verb| git init |

    本地初始化后,如要连接网络远程仓库可以再用

    \verb| git remote add |
  \end{enumerate}

  \item 工作区 -> 暂存区:add添加(也叫stage)
    \begin{enumerate}[A.]
      \item 保存修改并添加文件到暂存区(index,或叫staging area)

      \verb| git add <文件名> |

      常用简化命令如下:
        \begin{itemize}[\textbullet]
          \item \verb| git add -A % 所有更改 |
          \item \verb| git add .  % 新增、修改,不含删除 |
          \item \verb| git add -u % 修改、删除,不含新增 |
        \end{itemize}

      \item 或者查看区别

      \verb!git diff!
    \end{enumerate}

  \item 暂存区 -> 版本库:commit提交

  \verb| git commit -a -m "message" |

  其中,-a 表示全部 -m 添加一个留言。
  \begin{enumerate}[(a)]
    \item 如果弹出了vim截面,则用i命令进入编辑模式,输入完成后,esc退回到命令模式,
    \item 然后:wq保存并退出vim,即可回到git命令窗口。
  \end{enumerate}

  \item * 确认提交状况  git status

  \item 推送到远程仓库

  \verb| git push -branch |

  之前步骤1-5都是在本地的版本管理,直到这一步才牵扯。
\end{enumerate}

\subsection{总结章节样式}
\label{ssc:summary-title-formats}

实际上,章节样式划分为两类:
\begin{enumerate}[(1)]
  \item 章与各级节都是传统、常用的多层次数字编号,对应\LaTeX 层次 0 $\sim$ 3。
  \item 段落是新增的,其编号由全角括号包围,对应\LaTeX 层次 4 $\sim$ 5。
\end{enumerate}

\section{列表环境测试} \label{sec:list}

列表也是很常用的排版方式。
最方便的方法就是调用 enumerate 宏包定制标签,用 A a R r 1 分别代表大写字母、小写字母、大写罗马、小写罗马和阿拉伯数字编号。

为了偷懒和整齐,这次对一到三级列表的默认样式进行了设计。

\subsection{基本列表} \label{ssc:basiclist}

首先,先熟悉一下基本列表环境。
最常用的是有序列表。

\begin{enumerate}
  \item 中文 gb7714-2015.bbx
  \item gb7714-2015.bbx
  \item 中文 gb7714-2015ay.bbx
  \item gb7714-2015ay.bbx
\end{enumerate}

也挺常用的无序列表。

\begin{itemize}
  \item 中文 gb7714-2015.bbx
  \item gb7714-2015.bbx
  \item 中文 gb7714-2015ay.bbx
  \item gb7714-2015ay.bbx
\end{itemize}

本次修改的发起者,用description列表展示。
感觉这是一种不怎么常用的列表\footnote{大概是因为 word 中没有吧哈哈哈}。

\begin{description}
  \item[CNchence] 在github.com CNchence
  \item[marquistj13] 在github.com marquistj13
  \item[Wenda] 在github.com Williamwenda
  \item[CNchence] 在github.com CNchence
  \item[Wenda] 在github.com Williamwenda
\end{description}

更常见的是嵌套列表。

\subsection{嵌套}

先填充一些废话。\zhlipsum[6]

可以参见第~\ref{ssc:git-workflow}节的例子,也可以看下面这个。
区别在于,本小节的例子完全展示了\emph{默认参数}情况下的样式。

首先是有序列表。
\begin{enumerate}
  \item 介绍了时间序列分析的相关概念与理论。
  \item 重点对时间序列的平稳性特点、检验方法及平稳模型体系做了详细阐述。
  \begin{enumerate}
    \item 为充分反映出监测序列的季节周期性,建立了一套基于单测点监测序列的季节 ARIMA 模型分析方法。
    \item 对建模过程中的平稳性检验、模型建立步骤、定阶、参数估计以及检验进行了系统的阐述,
  \end{enumerate}
  \item 并对玉峰大桥某监测测点的实测数据进行了季节 ARIMA 建模与预测分析。
  \item 结果表明季节 ARIMA 模型能较好地模拟监测序列的变化趋势,预测精度较高。
  \begin{enumerate}
    \item 用力地嵌套列表。
    \begin{enumerate}
      \item 针对监测序列中某些测点间具有较高的相似性,
      \item 探讨了滞后协整分析的参数估计、模型检验等方法,
    \end{enumerate}
    \item 最后建立了支座位移与温度间的 ADL 与 ECM 模型,
    \item 结果表明该模型具有良好的拟合精度与预测效果。
  \end{enumerate}
\end{enumerate}

然后是无序列表的。
\begin{itemize}
  \item 介绍了时间序列分析的相关概念与理论。
  \item 重点对时间序列的平稳性特点、检验方法及平稳模型体系做了详细阐述。
  \begin{itemize}
    \item 为充分反映出监测序列的季节周期性,建立了一套基于单测点监测序列的季节 ARIMA 模型分析方法。
    \item 对建模过程中的平稳性检验、模型建立步骤、定阶、参数估计以及检验进行了系统的阐述,
  \end{itemize}
  \item 并对玉峰大桥某监测测点的实测数据进行了季节 ARIMA 建模与预测分析。
  \item 结果表明季节 ARIMA 模型能较好地模拟监测序列的变化趋势,预测精度较高。
  \begin{itemize}
    \item 用力地嵌套列表。
    \begin{itemize}
      \item 针对监测序列中某些测点间具有较高的相似性,
      \item 探讨了滞后协整分析的参数估计、模型检验等方法,
    \end{itemize}
    \item 最后建立了支座位移与温度间的 ADL 与 ECM 模型,
    \item 结果表明该模型具有良好的拟合精度与预测效果。
  \end{itemize}
\end{itemize}

下面是一个description列表,不知道中文叫什么。

\begin{description}
  \item [linxdcn] (在github.com linxdcn TongjiThesis)同学汇总的wildwolf、
  \item [svandex] 在github.com svandex masthesis)、
  \begin{description}
    \item [zhao-chen] 在github.com zhao-chen TongjiThesis)的版本。
    \item [收藏有本硕博模板的zhouyuan版] 在github.com zhouyuan/tongjithesis)。
  \end{description}
\end{description}

\zhlipsum[33]

下面是行内列表的展示。

\begin{inline}[1)]
  \item 中文 gb7714-2015.bbx
  \item gb7714-2015.bbx
  \item 中文 gb7714-2015ay.bbx
  \item gb7714-2015ay.bbx
  \item 中文 gb7714-2015.bbx
  \item gb7714-2015.bbx
  \item 中文 gb7714-2015ay.bbx
  \item gb7714-2015ay.bbx
\end{inline}

\section{表格测试} \label{sec:tabletest}

常用的表格是三线表。

\begin{table}[htb]
\centering
\caption{这不是个图}
\label{tab:fig}
\begin{tabular}[c]{lll}
  \toprule[1.5pt]
  0 & 1 & 2 \\ % title row
  \midrule[1pt]
  0 & 1 & 2 \\ % title row
  0 & 1 & 2 \\ % title row
  \bottomrule[1.5pt]
\end{tabular}
\end{table}

先填充一些废话。\zhlipsum[7]

在这里学习一下新定义的列形式。

\begin{table}[htb]
\centering
\caption{tabularx}
\label{tab:tabularx}
\begin{tabularx}{0.7\textwidth}{>{\raggedleft\arraybackslash}X>{\raggedleft\arraybackslash}X>{\centering\arraybackslash}X}
  \toprule[1.5pt]
  0 & 1 & 2 \\ % title row
  \midrule[1pt]
  0 & 1 & 2 \\ % title row
  0 & 1 & 2 \\ % title row
  \bottomrule[1.5pt]
\end{tabularx}
\end{table}

\section{代码环境}
\label{sec:code}

常用的代码展示,基于 Listings 宏包。

\subsection{Listings 环境展示}
\label{ssc:listings}

\begin{lstlisting}[style=monocolor,
  caption={first lst env},
  label={no label},
  name=测试第一个代码,
  language=PythonPlus]
  # import lxml
  import requests
  from bs4 import BeautifulSoup
  import re
\end{lstlisting}

先填充一些废话。\zhlipsum[8]

\begin{lstlisting}[language=ParamML,
  % caption=,
  % label=lst:,
  ]
<O  N="C4000Psi"  T="Material"  D="Concrete">
  <P N="d"  V="0.0000002248"  D="Density"/>
  <P N="E"  V="3604.9965"  D="Modulus of Elasticity"/>
  <P N="Fc28"  V="4"  D="Concrete Compressive Strength"/>
</O>
<!-- comment -->
\end{lstlisting}

不同于tabled 表~\ref{tab:fig},源代码的展示用\ref{lst:inp}不会显示开头。
好吧,是我错了。都是需要手动指定其是“图”、“表”、“代码”或“章节”。\textbackslash ref 命令只显示编号。
仔细一想,这样也好,只做最小程度的工作,格式完全可以 \textbackslash newcommand 或者用 VSCode 脚本辅助。

\begin{lstlisting}[language=Python,
  caption=Second CODE Block,
  label=lst:2nd,
  style=colored]
  import numpy as np
  def incmatrix(genl1,genl2):
      m = len(genl1)
      #compute the bitwise xor matrix
      M1 = bitxormatrix(genl1)
      for i in range(m-1):
          for j in range(i+1, m):
              [r,c] = np.where(M2 == M1[i,j])
              for k in range(len(r)):
                  VT[(i)*n + r[k]] = 1;

                  if M is None:
                      M = np.copy(VT)
                  else:
                      M = np.concatenate((M, VT), 1)
      return M
\end{lstlisting}

为 宏包 Listings 定义了许多种语言和样式,比如XML,再基于XML定义了 ParamML。
语法比较简单,看看 \tvt{tongjithesis.sty} 文件 也就懂了。
故不赘述。

第~\ref{sec:code} 小节的内容也被引用了。

\begin{lstlisting}[language=Python,
  caption=中文标题没有括号,
  label=lst:outer,
  style=colorEX]
  import numpy as np
  def incmatrix(genl1,genl2):
      m = len(genl1)
      #compute the bitwise xor matrix
      M1 = bitxormatrix(genl1)
      for i in range(m-1):
          for j in range(i+1, m):
              [r,c] = np.where(M2 == M1[i,j])
              for k in range(len(r)):
                  VT[(i)*n + r[k]] = 1;
      return M
\end{lstlisting}

还可以直接导入一个外部文件,并摘选其中的部分行。

\lstinputlisting[language=Python,
  caption=inputted Python,
  label=lst:inp,
  style=monocolor,
  ]{../code/tmp.py}

先填充一些废话。\zhlipsum[12]

\subsubsection{代码片段 Listings 引用}

这里测试三个代码片段的引用。

代码片段,没有前缀,直接中括号,见 \ref{no label} 代码。

代码片段,是按照现有的snippet直接插入的样式 \ref{lst:2nd} 代码。

代码片段, \ref{lst:outer} 代码,手动在环境之后加了label,可以被 vscode 插件检测到。

代码片段, \ref{lst:parenthesis} 代码,仅供测试,caption 加了花括号。

貌似不能从ref命令中自动调用。难道需要手动加 label 更好吗?
\begin{lstlisting}[language=Python,
  caption={Must have Parenthesis},
  label=lst:parenthesis]
  import numpy as np
  def incmatrix(genl1,genl2):
      m = len(genl1)
      #compute the bitwise xor matrix
      M1 = bitxormatrix(genl1)
      for i in range(m-1):
          for j in range(i+1, m):
              [r,c] = np.where(M2 == M1[i,j])
              for k in range(len(r)):
                  VT[(i)*n + r[k]] = 1;
      return M
\end{lstlisting}
\subsection{算法}
\label{ssc:algo}

比代码更加抽象的方式就是算法展示了。

先填充一些废话。\zhlipsum[7]

\begin{minipage}{0.75\textwidth}

\begin{algorithm}[H]
  \caption{测试算法}
  \label{alg:test}
    \begin{algorithmic}[1]
      \REQUIRE PDf
      \ENSURE xelatex biber xelatex*2
      \IF{some condition is true}
      \STATE do some processing
      \ELSIF{some other condition is true}
      \STATE do some different processing
      \ELSIF{some even more bizarre condition is met}
      \STATE do something else
      \ELSE
      \STATE do the default actions
      \ENDIF
    \end{algorithmic}
  \end{algorithm}

\end{minipage}
\newline

但是好像用上的机会不多。

algorithm 是个 float 环境,相当于 table。
其宽度用 minipage 可以调整宽度,但是会取消 \textbackslash intextsep 导致上下间距过小。

\begin{enumerate}[(1)]
  \item 偷懒的方案是手动加竖向间距。
  \item 更好的方案是用另一个可以接受宽度的的浮动体包围、algorithm 以选项 H 放置其中。
  比如 tcolorbox、fullwidth 等。
\end{enumerate}

\section{特殊标记方式}
\label{sec:mark-methods}

先填充一些废话。\zhlipsum[9]

\subsection{下划线测试}
\label{ssc:underline}

这是默认的\underline{下划线功能}的效果。
可以看到,比 MS Word 的下划线好看多了,不会产生文字底线与下划线重合的问题。

\subsection{行内原样抄写并带有双引号命令}
\label{ssc:tvt}

考虑到会出现一些后缀名或者文件名之类的内容,所以新添了一个命令:
\verb|\tvt{}|,
其作用是在一个原样抄写的文字外围加上双引号。

CIS/2标准使用\tvt{*.stp}中性格式文件保存数据模型的数据。
对应上述梁的大纲,该梁在\texttt{*.stp}中性格式文件中的主要内容如下。

\subsection{行内原样抄写并带有圆括号命令}
\label{ssc:pvp}

同理,如果不想两端不用引号(quotation marks),可以用圆括号(parentheses)的版本。
\verb|\pvp{}|
\footnote{已经有了 TvT 了,怎么能没有 PvP 呢?来自星际二的梗。}
。

名称属性 \pvp{N} 必须是文本格式、且不允许空格或特殊字符。
值属性 \pvp{V} 可以是常数、文本、函数表达式,或者引用其他的 对象元素。

\subsection{交叉引用}
\label{ssc:cross-ref-cmd}

先填充一些废话。\zhlipsum[10]

交叉引用命令 \textbackslash ref 是非常有用的,但是格式上有一个注意点:
最好在命令的前后用“不可分割空格”(即$\sim$)连接关键文字,以避免换行。
比如\tvt{图 ~ \textbackslash 1.1 ~}

本次添加了关于“图”、“表”、“章节”等常用的交叉引用命令,方便调用。

然后测试一下新添加的交叉引用命令。
\begin{itemize}[\textbullet]
  \item \reftab{tab:tabularx}
  \item \reffig{fig:ken}
  \item \refalg{alg:test}
  \item \reflst{lst:outer}
  \item \refequ{eq:B}
  \item \refsec{ssc:cross-ref-cmd}
  \item \refcha{cha:intro}
\end{itemize}

好处是,VS Code 依然可以识别这些命令是交叉引用,所以括号内的自动补全是 label 没错。

但是,这种命令还不如直接用 snippet 完成呢,snippet 更具有灵活性。
所以,未来的版本可能会抛弃这个功能吧。


\end{document}
