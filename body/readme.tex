% !TeX root = ../Main/thesis.tex
% !TEX program = xelatex
\documentclass[../Main/thesis.tex]{subfiles}
\begin{document}

\chapter{同济 \LaTeX 大论文模板}

为了\textbf{简化大论文的排版工作},决心采用LaTeX进行论文撰写。 感谢校友
marquistj13 的
\href{https://github.com/marquistj13/TongjiThesis}{TongjiThesis 模版}。

本 README 记录了自己在 \LaTeX{} 方面踩过的坑,以及按照自己的思路进行的调整。

因为自己的 \LaTeX 知识水平太低,对主要宏包的运用可谓
NAIVE,身边也没有人能传授排版的经验,最终也只能做了一点微小的工作。

但学到了一项新技能也足以让人感到 EXCITED!

\section{主要改动 CHANGES}
\label{ux4e3bux8981ux6539ux52a8-changes}

因为模板调整与论文撰写同时进行,而自己的大论文放在了 private repo
中,所以这个仓库是粗暴地把我所采用样式相关文件复制了出来而已。
也即:丢失了调整过程的步骤,没有 commit 历史可供参考。

\begin{quote}
以下称同济大学研究生院的《同济大学学位论文写作示例(2017)》与《同济大学研究生学位论文写作规范(试行)(2017)》为``校方模版'';
称 marquistj13 的 TongjiThesis 模版为``原模版''。
\end{quote}

本节总述了本模板相较于原模版的修改。
部分修改较繁琐、或需要展开说明,则在后续独立一节详述。

\subsection{手动安装模板文件}

手动安装了 \texttt{.cls}、\texttt{.sty}
等模板所用的文件,从而方便了分章节编译、方便了多项目使用同一模板。

主要步骤是: 1. 用 cmd 命令 \texttt{kpsewhich\ -\/-var-value=TEXMFHOME}
得知路径 \texttt{X:/xxx/texmf} 2. 在其中继续创建 3 层文件夹
\texttt{tex/latex/tongjithesis/}, 3. 把 \texttt{.cls} 和 \texttt{.sty}
文件放进文件夹 \texttt{tongjithesis/}。

具体参见
\protect\hyperlink{ux5cux25e5ux5cux25a4ux5cux2584ux5cux25e7ux5cux2590ux5cux2586ux5cux25e6ux5cux2594ux5cux25b9ux5cux25e8ux5cux2589ux5cux25af}{处理(改良)}
一节。

\subsection{封面布局格式调整}

微调了封面的布局。 但是校方模板中也有不少含糊,所以最终效果必有纰漏。

\begin{enumerate}
\def\labelenumi{\arabic{enumi}.}
\item
  校方模版中校徽都不舍得提供向量图。

  借用了
  \href{https://github.com/marquistj13/TongjiThesis/issues/20}{关于Logo
  \#20} 中 \href{https://github.com/chennanzhang}{chennanzhang} 用宏包
  \texttt{Tikz} 绘制的学校 Logo。

  不过,其中的距离尺寸都是手动测量的,与校方模板终归有偏差。 原
  \texttt{\textbackslash{}includegraphics} 仍予以保留。
\item
  各个元素之间的间距没有出确定值,只是大概的``换一行''、``空半行''。
\item
  封面内容的行距被 Word
  的``对齐到网格''控制,导致展示的样式与本模板有偏差。

  比如申请人个人信息表格的行距明显不是单倍,根据目测调整为 1.75 倍。
\item
  申请人个人信息表格要求``左侧缩进 4.5
  字符'',竟然是相对于一个文本框、并不是页面位置。
  于是也只能通过目测,设置表格为居中。
\item
  申请人个人信息表格首列分散对齐,采用了 \texttt{array} 宏包的
  \texttt{W\{s\}\{width\}} 生成一个相当于
  \texttt{\textbackslash{}makebox} 的列。

  \begin{itemize}
  \item
    分散对齐,普遍采用
    \texttt{\textbackslash{}makebox{[}width{]}{[}s{]}\{...text...\}}
    命令。
  \item
    删除了原本的 \texttt{\textbackslash{}tongji@put@title\{\}} 方案。

    \begin{itemize}
    \item
      每一行重复出现,显得臃肿,故类比添加数学环境到列的方法,将此命令添加到列定义中,失败。
    \item
      初见令我混乱,比如 \texttt{\textbackslash{}hb@xt@} 语法生僻。
    \end{itemize}
  \item
    尝试 \texttt{ragged2e} 时,命令 \texttt{\textbackslash{}justifying}
    使得首列宽度错误,且此宏包已经无人维护,故放弃。
  \end{itemize}
\item
  从 \texttt{.cfg} 文件中删除了首列文字配置,删除了分隔符冒号。

  \begin{itemize}
  \item
    主要处于代码美观简介考虑,因为``姓名''、``学号''和``指导教师''等文字并不会复用,所以没必要用变量替代。
  \item
    冒号作为分隔符,可以在列定义的时候用 \texttt{\textgreater{}\{:\}l}
    语法插入右列,不需要单独一列 \texttt{c} 放置。

    \begin{itemize}
    \item
      插入左列(首列)会导致列宽不等,没有深究原因。
    \end{itemize}
  \end{itemize}
\end{enumerate}

所以,封面的排版完全依赖自己的朴素的审美观,不能保证符合要求。
但起码,已经尽力使得中英文封面对应对象对齐。

在新的校方模板中,封面 \texttt{electronic} 和 \texttt{secret}
选项均已失去了意义; 但考虑到日后万一又加了回来,故不做删除、予以保留。

此外,尝试采用\texttt{CTeX} 提供的 \texttt{\textbackslash{}zihao\{\}}
命令,但是发现反而没有原模版的旧命令好用。
因为校方模版许多地方要求同时更改字号和行距,旧命令可以很方便地执行。

\subsection{封面与书脊页}

把中文封面、英文封面和书脊统一到一个新命令
\texttt{\textbackslash{}makecover} 中。 它的必需参数就是
\texttt{cover.tex}
的文件位置,并且接受一个星号(asterisk)以控制是否生成书脊。

\texttt{\textbackslash{}makecover\{../pages/cover\}}

中英文封面加上原创性声明和授权书页面,在 \texttt{frontmatter}
之前,页面布局由 \texttt{\textbackslash{}AtBeginDocument\{\}} 中的
\texttt{\textbackslash{}pagestyle\{tongji@empty\}} 和
\texttt{\textbackslash{}pagenumbering\{roman\}} 控制,即在 PDF
中采用\textbf{小写罗马字母}页码(文件中不会打印编号,只有电子书签)。

从而,可以与它们之后的 \texttt{\textbackslash{}frontmatter}
(含中文摘要、英文摘要和目录)的\textbf{大写罗马字母}页码有所区分。

为中文封面和英文封面添加了 PDF 书签。

\subsection{书脊页}

新 \texttt{\textbackslash{}bookspine}
命令生成形如校方模版的带边框的书脊页面,并且放到
\texttt{\textbackslash{}makecover} 中,可被自动调用。 边框
\texttt{\textbackslash{}framebox} 内的盒子
\texttt{\textbackslash{}makebox} 尺寸不宜撑满
\texttt{\textbackslash{}textheight},因为若如此做,考虑到 frame
的尺寸,会在书脊页之前产生额外的空白页面。

原 \texttt{\textbackslash{}shuji}
命令不删除,因为要用它生成\textbf{真}书脊(没有边框的样式)。
因为没有边框,所以也不需要用 \texttt{\textbackslash{}newgeometry}
生成页边距很小的版面了。

特别地,书脊中使用了字形旋转机制。
然而,最初在排版时就发现,旋转后的基线对齐是有问题的。
实际上,这个问题是 \texttt{XeLaTeX} 的底层问题。


\begin{itemize}
\item
  \href{https://sourceforge.net/p/xetex/bugs/164/}{Adjust position of
  vertical text for better alignment}
\item
  \href{https://github.com/CTeX-org/forum/issues/93}{XeTeX
  是如何旋转字形的,是否实现有误?}
\end{itemize}

通过目测,认为偏移大概是半个字符; 现在可以得知,汉字旋转后,以行基线
baseline 居中对齐。 所以,手动加入
\texttt{\textbackslash{}hspace\{0.5em\}} 使其居中。
然而,经过测量发现,对齐还是有问题。

经过测试,\texttt{\textbackslash{}hspace\{0.65em\}} 始得汉字基本处于
framebox 的中央。 猜测是跟仿宋字体的宽高比 \(0.7\) 有关。 四号字高度是
\(14.0\) pt(\(1.00\) em),即 frame 的横向尺寸。 四号字显示宽度为
\(0.7 \times 14 = 9.8\) pt(\(0.70\) em),所以反转后基线左、右侧各
\(4.9\) pt。 则汉字轴线到右侧 frame 的距离依然是四号字高度 \(14.0\) pt。
则为了使得汉字轴线在 frame 居中,左侧需要填充的偏移距离是
\(14.0-4.9=9.1\) pt(\(0.65\) em)。

不过,如果同时出现旋转和非旋转的字体,那么这个偏移问题还是没有解决。
最明显的例子,就是标题中的汉字旋转后,与英文字母或模板中的 \LaTeX 符号,有明显的上下偏移。 需要独立调用书脊的命令
\texttt{\textbackslash{}bookspine{[}{]}}
输入标题参数(定义时可选参数要用 \texttt{{[}{]}} 输入),并且用
\texttt{\textbackslash{}raisebox\{-0.35em\}\{eng\ text\}}
手动移动英文部分的对齐。

\begin{quote}
基线与底线的偏移量与字体相关。 比如中易宋体 SimSun 基线与底线偏移
\(36/256\),思源宋体 Source Sans 基线与底线偏移 \(120/1000\)。
\end{quote}

\subsection{声明页}

原创性声明和授权书页面从 \texttt{\textbackslash{}makecover}
中移除,现在需要手动执行。

并利用宏包 \texttt{\{pdfpages\}} 使其可以接收一个 PDF 扫描件替换电子版。

\begin{itemize}
\item
  \texttt{\textbackslash{}makeauthorizationpage{[}../pages/scan.pdf{]}\%\ 可导入扫描页}
\item
  \texttt{\textbackslash{}makedeclarepage\ 不导入扫描页则生成电子版}
\end{itemize}

\begin{Shaded}
\begin{Highlighting}[]
\FunctionTok{\textbackslash{}makeauthorizationpage}\CommentTok{% 生成电子版授权书}
\FunctionTok{\textbackslash{}makedeclarepage}\NormalTok{[../data/scandecl.pdf]}\CommentTok{% 插入扫描版的《原创声明》PDF文件}
\end{Highlighting}
\end{Shaded}

\subsection{PDF 元数据}

用 \texttt{\textbackslash{}hypersetup\{\textbackslash{}pdftitle=\{\}\}}
把文档的元数据写入 PDF 文件的属性。

标题、作者等元数据都写在 \texttt{cover.tex} 之中、在
\texttt{\textbackslash{}makecover} 时读取,所以在此之后才能执行 PDF
元数据写入。

\subsection{紧凑布局模式}

仅为了自己写大纲用的,创建了一套更加紧凑的布局。 原本命名为
\texttt{draft} 草稿选项,但是发现这是个冲突的选项。 考虑到
\texttt{electronic} 也是个没有用的选项,于是就合并了。

具体使用方法是: 1. 开启文档类型中的一个新增的参数
\texttt{electronic=true},它会使得原本的多处
\texttt{\textbackslash{}cleardoublepage} 变为
\texttt{\textbackslash{}clearpage}。 2. 使用
\texttt{\textbackslash{}makecover*\{\}} 则不生成书脊,或覆盖封面页面
\texttt{\textbackslash{}begin\{titlepage\}\textbackslash{}chncover\textbackslash{}end\{titlepage\}}
连英文封面也不产生。 3.
不调用\texttt{\textbackslash{}makeauthorizationpage} 或
\texttt{\textbackslash{}makedeclarepage},即不生成声明页面。 4.
后续部分根据要求取舍,比如索引、致谢等。

\begin{quote}
这些特殊页面的生成命令的解耦就是为了能够更加自由的控制。
当然,代价就是新使用者可能会觉得麻烦。
\end{quote}

\subsection{索引修复}

修复了索引命令(\texttt{\textbackslash{}listof...})加星号(asterisk)之后,仍然出现在目录页面的错误。

\subsection{交叉引用命令}

交叉引用时,前缀与编号之间通常建议用不中断空格
\texttt{\textasciitilde{}}
连接,以避免不良的分行,打断了``图''、``表''、``第''等前缀词、编号数字和可能的后缀``节''、``章''。

\begin{Shaded}
\begin{Highlighting}[]
\CommentTok{% 交叉引用的命令}
\FunctionTok{\textbackslash{}newcommand*}\NormalTok{\{}\ExtensionTok{\textbackslash{}reftab}\NormalTok{\}[1]\{}\FunctionTok{\textbackslash{}tablename}\NormalTok{~}\KeywordTok{\textbackslash{}ref}\NormalTok{\{}\ExtensionTok{#1}\NormalTok{\}\}}
\FunctionTok{\textbackslash{}newcommand*}\NormalTok{\{}\ExtensionTok{\textbackslash{}reffig}\NormalTok{\}[1]\{}\FunctionTok{\textbackslash{}figurename}\NormalTok{~}\KeywordTok{\textbackslash{}ref}\NormalTok{\{}\ExtensionTok{#1}\NormalTok{\}\}}
\FunctionTok{\textbackslash{}newcommand*}\NormalTok{\{}\ExtensionTok{\textbackslash{}refalg}\NormalTok{\}[1]\{}\FunctionTok{\textbackslash{}ALG@name}\NormalTok{~}\KeywordTok{\textbackslash{}ref}\NormalTok{\{}\ExtensionTok{#1}\NormalTok{\}\}}
\FunctionTok{\textbackslash{}newcommand*}\NormalTok{\{}\ExtensionTok{\textbackslash{}reflst}\NormalTok{\}[1]\{}\FunctionTok{\textbackslash{}lstlistingname}\NormalTok{~}\KeywordTok{\textbackslash{}ref}\NormalTok{\{}\ExtensionTok{#1}\NormalTok{\}\}}
\FunctionTok{\textbackslash{}newcommand*}\NormalTok{\{}\ExtensionTok{\textbackslash{}refequ}\NormalTok{\}[1]\{}\FunctionTok{\textbackslash{}equationname}\NormalTok{(}\KeywordTok{\textbackslash{}ref}\NormalTok{\{}\ExtensionTok{#1}\NormalTok{\})\}}
\FunctionTok{\textbackslash{}newcommand*}\NormalTok{\{}\ExtensionTok{\textbackslash{}refsec}\NormalTok{\}[1]\{第~}\KeywordTok{\textbackslash{}ref}\NormalTok{\{}\ExtensionTok{#1}\NormalTok{\}~节\}}
\FunctionTok{\textbackslash{}newcommand*}\NormalTok{\{}\ExtensionTok{\textbackslash{}refcha}\NormalTok{\}[1]\{第~}\KeywordTok{\textbackslash{}ref}\NormalTok{\{}\ExtensionTok{#1}\NormalTok{\}~章\}}
\end{Highlighting}
\end{Shaded}

比如说,\texttt{见\textbackslash{}refsec\{\}。}
可以直接生成``见第\textsubscript{X.X}节。''这样的语句。 而且,好处是 VS
Code 依然识别这个命令是 \texttt{\textbackslash{}ref},可以自动联想到
label。

为了配合它的使用,还在 VS Code 中制定了一个 \texttt{tjref} 的 snippet。

\begin{Shaded}
\begin{Highlighting}[]
\ErrorTok{"TJ} \ErrorTok{Ref} \ErrorTok{with} \ErrorTok{Prefix":} \FunctionTok{\{}
  \DataTypeTok{"prefix"}\FunctionTok{:} \StringTok{"tjref"}\FunctionTok{,}
  \DataTypeTok{"body"}\FunctionTok{:} \OtherTok{[}
    \StringTok{"}\CharTok{\textbackslash{}\textbackslash{}}\StringTok{ref$\{1|tab,fig,sec,cha,alg,lst,equ|\}\{$1:$2\}"}
  \OtherTok{]}\FunctionTok{,}
  \DataTypeTok{"description"}\FunctionTok{:} \StringTok{"TJ Ref with Prefix"}
\FunctionTok{\}}\ErrorTok{,}
\end{Highlighting}
\end{Shaded}

但是此时发现,这个功能直接用 snippet 完成仿佛更加合适、灵巧。
因此又放弃了这些命令。

附上取而代之的 snippet。

\begin{Shaded}
\begin{Highlighting}[]
\ErrorTok{"Ref} \ErrorTok{with} \ErrorTok{Prefix":} \FunctionTok{\{}
    \DataTypeTok{"prefix"}\FunctionTok{:} \StringTok{"pref"}\FunctionTok{,}
    \DataTypeTok{"body"}\FunctionTok{:} \OtherTok{[}
        \StringTok{"$\{1|第,图,表,公式,算法,代码|\}~}\CharTok{\textbackslash{}\textbackslash{}}\StringTok{ref\{$0\}$\{2|~,所示,~节,~章|\}"}
    \OtherTok{]}\FunctionTok{,}
    \DataTypeTok{"description"}\FunctionTok{:} \StringTok{"Ref with Prefix"}
\FunctionTok{\}}\ErrorTok{,}
\end{Highlighting}
\end{Shaded}

\subsection{模板结构调整}

按照
\href{http://texdoc.net/texmf-dist/doc/latex/base/clsguide.pdf}{clsguide}
的理解,宏包 package 是更加具有通用性的文件,而 cls
类文件仅仅对一类文档有用。 所以,按照自己的思路重新调整了cls和sty文件。

\begin{itemize}
\item
  \texttt{.cls} 旨在满足\textbf{基本要求}:

  《同济大学学位论文写作规范》等校方模版的要求,学术论文习惯(例如三线表格)。
\item
  \texttt{.sty} 提供\textbf{扩展工具}:

  此类并不是每个人都需要,完全可以注释、删除掉。
\end{itemize}

比如原本 \texttt{.cls} 中有细致地美化了脚注
\texttt{\textbackslash{}footnote} 命令。
因为下划线、脚注并不常用,而且校方模版中也没有样式说明,所以移入了
\texttt{.sty} 文件。

同理,表格类宏包和设置(比如跨页长表格)基本都从 \texttt{.cls} 移至
\texttt{.sty}; \texttt{.cls} 文件中只有基本的三线表格实现。
特殊环境(算法环境,源代码环境)也放入了 \texttt{.sty} 文件。

实际上,本模板的 \texttt{tongjithesis.sty} 文件完全可以祛除。

尽量将相关的设置集中在一处,便于调整。

注释掉若干没有工作的宏包。在确定其工作机理和生效位置前,暂不删除。

删除了部分过时的注释。

\subsection{章节样式设置}

出于个人审美,章节名称中的数字编号比汉字小一号。

\begin{quote}
以前 CAD 课的要求``数字要比汉字小一号'',算是一个纪念画图日子的小彩蛋。
\end{quote}

章节名称中的数字字体可利用宏包 \texttt{\{fontspec\}} 控制,将原
\texttt{romantitle} 布尔型选项改为可以三选一的 \texttt{titlenum} 选项:
* \texttt{\textbackslash{}rmfamily} * \texttt{\textbackslash{}sffamily}
* \texttt{SimHei}

具体可见
\protect\hyperlink{ux5cux25E7ux5cux25ABux5cux25A0ux5cux25E8ux5cux258Aux5cux2582ux5cux25E6ux5cux25A0ux5cux2587ux5cux25E9ux5cux25A2ux5cux2598ux5cux25E6ux5cux25A0ux5cux25B7ux5cux25E5ux5cux25BCux5cux258Fux5cux25E8ux5cux25AEux5cux25BEux5cux25E7ux5cux25BDux5cux25AE}{章节标题样式设置}
一节的内容。

开启了 \texttt{\textbackslash{}paragraph} 和
\texttt{\textbackslash{}subparagraph} 即 4、5 级层次。

\subsection{优化 PDF 电子书签}

用 \texttt{bookmark} 宏包提供了更方便地生成电子书签。 相比原模版的
\texttt{hyperref} 宏包方法,主要优势在于: 1.
可以避免跳转不准的问题,即不需要
\texttt{\textbackslash{}phantomsection}。 2.
只需要编译一次也能产生正确的电子书签。

值得一提,完整的文档依然需要两次编译,才能: 1. 生成目录; 2. 写入正确的
PDF
页码,即封面小写罗马字母、其他前文大写罗马字母、正文和后文阿拉伯数字。

\subsection{URL 链接的样式}

通常输入网址的方法有: 1. 默认启用 \texttt{url}
宏包,命令\texttt{\textbackslash{}url\{URL\}}
直接显示网址,单击打开该网址。 2. 启用 \texttt{hyperref}
宏包,命令\texttt{\textbackslash{}href\{URL\}\{text\}} 显示
text,但是单机之后打开 URL 网址。

前者可以在导言区定义 \texttt{\textbackslash{}urlstyle\{xx\}},其中
\texttt{xx} 从以下四种默认样式中选择,仅仅更改字体。 * tt 使用等宽字体 *
rm 使用衬线字体 * sf 使用非衬线字体 * same 跟上文字体保持一致

如果不满足于仅仅修改字体,可以用命令
\texttt{\textbackslash{}def\textbackslash{}UrlFont\{\textbackslash{}ttfamily\}}
设置更多样式

后者的样式定义,则在 \texttt{\{text\}} 中指定即可。

本模板新定义了命令 \texttt{\textbackslash{}colorurl},用于展示 URL。

\subsection{圆圈序号}

之前一直困于多级列表与段落的样式过于重合,想寻找中文中常用的、更多的序号样式。
整理脚注的时候,也发现圆圈样式的实现很特别,所以就找到了一篇关于
\href{https://stone-zeng.github.io/2019-02-09-circled-numbers/}{带圈数字}
的实现教程,将部分成果吸纳进了这个模板。 \textgreater{}
也是为了跟老板编的教材保持一致。

\begin{enumerate}
\def\labelenumi{\arabic{enumi}.}
\item
  传统方法是利用 \texttt{pifont} 宏包,它封装了很多PostScript
  字体。这也是原模板采用的美化方法。
\item
  一个更加灵活的方法,是用大名鼎鼎的 \texttt{Tikz} 宏包绘制一个圆形。

  \begin{enumerate}
  \def\labelenumii{\arabic{enumii}.}
  \item
    最初找到的是
    \href{https://www.latexstudio.net/archives/1571.html}{LaTeX 工作室}
    的方案。
  \item
    更好的方案是
    \href{https://tex.stackexchange.com/questions/7032/good-way-to-make-textcircled-numbers\#}{StackExchange}
    的方案。相比前者,有两个优势:

    \begin{enumerate}
    \def\labelenumiii{\arabic{enumiii}.}
    \item
      用 \texttt{{[}baseline=(char.base){]}} 定位,而不是手动的
      \(0.7ex\) 上下调整。
    \item
      评论区建议了 \texttt{\textbackslash{}DeclareRobustCommand} 定义。
    \end{enumerate}
  \end{enumerate}
\end{enumerate}

\begin{Shaded}
\begin{Highlighting}[]
\CommentTok{% 方案 1 利用 Tikz}
\FunctionTok{\textbackslash{}newcommand*}\NormalTok{\{}\ExtensionTok{\textbackslash{}circled}\NormalTok{\}[1]\{}\FunctionTok{\textbackslash{}lower}\NormalTok{.7ex}\FunctionTok{\textbackslash{}hbox}\NormalTok{\{}\FunctionTok{\textbackslash{}tikz\textbackslash{}draw}\NormalTok{ (0pt, 0pt) circle (.5em) node \{}\FunctionTok{\textbackslash{}makebox}\NormalTok{[1em][c]\{}\FunctionTok{\textbackslash{}small}\NormalTok{ #1\}\};\}\}}
\FunctionTok{\textbackslash{}robustify}\NormalTok{\{}\FunctionTok{\textbackslash{}circled}\NormalTok{\}}\CommentTok{% 依赖于 pkg\{etoolbox\} 的命令}
\CommentTok{% 方案 2 利用 Tikz}
\FunctionTok{\textbackslash{}newcommand*\textbackslash{}circled}\NormalTok{[1]\{}\FunctionTok{\textbackslash{}tikz}\NormalTok{[baseline=(char.base)]\{}\FunctionTok{\textbackslash{}node}\NormalTok{[shape=circle,draw,inner sep=2pt] (char) \{#1\};\}\}}
\end{Highlighting}
\end{Shaded}

此外,还有提到用 \texttt{addfontfeatures\{Annotation=2\}}
等,但是没有成功。

最终的成果,运用到了 3 处: 1.
脚注的序号现在是一个三选一的命令了,可以选择 \textbf{pifont} 或者
\textbf{圆圈} 或者 \textbf{无修饰}。 2. subparagraph
的序号现在是圆圈数字了。 3. 行内列表的编号。

\subsection{列表样式}

利用 \texttt{\{enumitem\}}
宏包预定义了有序列表、无序列表和关键词列表的样式,包括缩进、标签样式和对齐。
还定义了两种行内列表,即不会换行的水平列表。

\begin{itemize}
\item
  inline 是用 ``(1)''进行编号的。
\item
  inlinecn 是用``第一''进行编号的。

  \begin{itemize}
  \item
    用到了 \texttt{ctex} 的一个汉化特性命令
    \texttt{\textbackslash{}chinese},从而将数字以中文数字形式输出。
  \end{itemize}
\end{itemize}

特别地,简历页面有序列表形式预设为 \texttt{{[}1{]}.} 。

\subsection{表格样式}

默认的表格内容环境是 \texttt{tabular},其大小是根据内容变化的。

如果表格较多的情况下,统一表格尺寸可能更加美观。原生命令是
\texttt{tabular*} 环境,然而其功能较弱。

所以,用宏包 \texttt{\{tabularx\}} 的 \texttt{tabularx}
表格环境实现固定宽度的表格。 其提供了 \texttt{l\ c\ r\ p} 之外的对齐样式
\texttt{X},会平均分配列宽,列内默认左对齐。

此外,利用宏包 \texttt{\{array\}} 在 \texttt{.sty}
中新定义了三种列对齐方式: * \texttt{a\{1\}} 左对齐 * \texttt{s\{1\}}
居中 * \texttt{d\{1\}} 右对齐

\begin{quote}
记忆方法:左手键盘按键的左、中、右方向。
\end{quote}

这三种对齐方式的必需参数为列宽分配权重。 例如
\texttt{\textbackslash{}begin\{tabularx\}\{16cm\}\{a\{0.75\}s\{1\}d\{1.25\}X\}}
意义是: 第一列左对齐,中列居中,第三列右对齐,且按照 \(3:4:5:4\)
的比例分配总计 \(16cm\) 的列宽。 *
注意,\textbf{权重之和必须等于总列数},比如该例中 \(0.75+1+1.25+1=4\)。

\subsection{横置页面}

宏包 \texttt{\{lscape\}} 和 \texttt{\{pdflscape\}}
可以产生内容横置的页面。区别是后者会把页面旋转,方便阅读。

此外,宏包 \texttt{\{rotating\}} 可以产生特定的图片或表格环境。

\subsection{下划线}

原生的下划线命令 \texttt{\textbackslash{}underline}
是不能换行的,很多时候不方便使用。 通常,采用 \texttt{\{ulem\}}
宏包提供的下划线更加使用,还有波浪线等多种样式可以选择。

在 \texttt{\{ulem\}} 宏包基础上,还有针对中文汉字的
\texttt{\{xeCJKfntef\}} 宏包可用。
它增加了对汉字的识别能力,可以在标点符号处自动断开。 原模版的
\texttt{\{CJKfntef\}} 已经失效。

\subsection{新增环境类型:算法(伪代码)}

采用宏包 \texttt{\{algorithms\}},它其实是两个宏包
\texttt{\{algorithm\}} 与 \texttt{\{algorithmic\}}
的合集,是最基础的算法排版宏包。

\begin{itemize}
\item
  前者类似于 \texttt{table}
  环境,控制算法框为一个浮动体,从而控制其页面位置、组织跨页、添加
  caption 和 label 等。
\item
  后者类似于 \texttt{tabular} 环境,生成算法内容。
\end{itemize}

宏包 \texttt{\{algorithm\}} 文档
\url{http://mirrors.rit.edu/CTAN/macros/latex/contrib/algorithms/algorithms.pdf}。

\subsection{新增环境类型:源代码}

用 \texttt{lstlisting} 环境展示源代码。

用 \texttt{\textbackslash{}lstset\{\}}
的方式统一设置了布局、对齐、边框和标题位置等通用格式。

用
\texttt{\textbackslash{}lstdefinelanguage\{NewLanguageName\}{[}dialect{]}\{BasedLanguage\}\{key=value\}}
的方法,扩展了 Python
的关键词列表,并据其继续包装了一个可供自定义的语言;同理,把
\texttt{OpenBrIM} 设定为一个语言。

可用 \texttt{lstdefinestyle=\{\}}
以值对的方式定义样式,即关键词、字符串等不同类型代码的高亮颜色、字体样式等:
1. 基础样式 \texttt{monocolor} 是只有黑白,依靠字体区分关键词; 2. 样式
\texttt{colored} 是基本的配色方案,可以不依靠宏包 \texttt{\{color\}}
使用,也可以把其中``注释''类型的绿色改为深绿色,以求美观; 3.
针对扩展过后的 Python 建成了一个相对色彩丰富的样式 \texttt{colorEX}。

\begin{quote}
不能把 \texttt{OpenBrIM} 建立在 XML 之上,否则 tag 不能正常显示
\texttt{identifierstyle} 样式。 原因是:\texttt{listings} 宏包的 XML
语言尚未完成。
\end{quote}

所以针对需要的 \texttt{OpenBrIM},特意完成了相关的环境定义: 1. 通用的
\texttt{XML} 语言,包括:

\begin{enumerate}
\def\labelenumi{\arabic{enumi}.}
\item
  tag 用 identifier 实现高亮,
\item
  attribute key 用\texttt{空格}和\texttt{=}作为前后边界定义,高亮形式为
  \texttt{keyworldstyle},
\item
  attribute value 定义为字符串,
\item
  text 信息定义为没有格式的字符串。
\item
  仿照 Firefox 浏览器的样式,定义了 \texttt{XML} 的高亮格式。

  参照
  \url{https://tex.stackexchange.com/questions/10255/xml-syntax-highlighting}
  。
\item
  专用 \texttt{ParamML} 语言。

  因为通用的 \texttt{XML}
  在高亮属性的键时,会把作为分界符的\texttt{=}一起高亮,效果不甚满意。
  所以,没有从 \texttt{XML} 继承定义
  \texttt{ParamML},而是重新定义了一种语言及其配套的高亮(仿照OpenBrIM平台样式)。
\end{enumerate}

重定义 \texttt{\textbackslash{}lstlistingname} 和
\texttt{\textbackslash{}lstlistlistingname} 以更改环境名。

用 \texttt{\textbackslash{}lstlistoflisting}
可以生成一个代码索引,类似于 \texttt{list\ of\ tables} 之类。 * 设置了
\texttt{nolol} 的代码不会进入索引。 * 可用下面的命令更改对齐和样式:
\texttt{\textbackslash{}renewcommand\textbackslash{}l@lstlisting{[}2{]}\{\textbackslash{}@dottedtocline\{1\}\{0em\}\{2em\}\{\textbackslash{}lstlistingname\textasciitilde{}\#1\}\{\#2\}\}}。
* 定制了 \texttt{listings} 索引页面的页眉、页脚样式。

值得一提的是,在代码块中设置 \texttt{name=} 而不设置
\texttt{caption=},则代码索引中就不出现这一项编号了、但仍保留 name。
或许代码不需要全部罗列,那么这可能是更好的解决方案------只有关键代码给出
name 并列入索引。

宏包 \texttt{\{Listings\}} 文档
\url{http://texdoc.net/texmf-dist/doc/latex/listings/listings.pdf}。

\subsection{页面布置}

学校的版面设置要求是``上、下 2.54cm,左、右 3.17cm,页眉、页脚
2.0cm,装订线 0 cm''。 然而,宏包 \texttt{\{geometry\}}
中的参数定义方式和 MS Office Word 有区别。

有两套处理思路:

\begin{itemize}
\item
  在原模板的基础上,微调尺寸参数。

\begin{Shaded}
\begin{Highlighting}[]
\DataTypeTok{ignoreall,}
\DataTypeTok{top}\OtherTok{=}\FloatTok{30.34}\StringTok{mm,}
\DataTypeTok{headsep}\OtherTok{=}\FloatTok{4.94}\StringTok{mm,}
\DataTypeTok{headheight}\OtherTok{=}\FloatTok{24.81}\StringTok{mm,}
\DataTypeTok{bottom}\OtherTok{=}\FloatTok{25.4}\StringTok{mm,}
\DataTypeTok{footskip}\OtherTok{=}\FloatTok{5.4}\StringTok{mm,}
\end{Highlighting}
\end{Shaded}
\item
  打开 \texttt{includehead} 选项。

\begin{Shaded}
\begin{Highlighting}[]
\DataTypeTok{includehead}\OtherTok{=}\KeywordTok{true}\StringTok{,}
\DataTypeTok{top}\OtherTok{=}\FloatTok{20.00}\StringTok{mm,}
\DataTypeTok{headheight}\OtherTok{=}\FloatTok{5.4}\StringTok{mm,}
\DataTypeTok{includefoot}\OtherTok{=}\KeywordTok{false}\StringTok{,}
\DataTypeTok{bottom}\OtherTok{=}\FloatTok{25.4}\StringTok{mm,}
\DataTypeTok{footskip}\OtherTok{=}\FloatTok{5.4}\StringTok{mm,}
\end{Highlighting}
\end{Shaded}
\end{itemize}

详细解释见
\protect\hyperlink{geometry-ux5cux25e9ux5cux25a1ux5cux25b5ux5cux25e9ux5cux259dux5cux25a2ux5cux25e5ux5cux25b0ux5cux25baux5cux25e5ux5cux25afux5cux25b8ux5cux25e9ux5cux25a1ux5cux25b5ux5cux25e7ux5cux259cux5cux2589ux5cux25e9ux5cux25a1ux5cux25b5ux5cux25e8ux5cux2584ux5cux259aux5cux25e5ux5cux25b0ux5cux25baux5cux25e5ux5cux25afux5cux25b8}{页面尺寸、页眉页脚尺寸}
一节。

\subsection{伪粗体字体复制产生乱码}

如果按照原模版中,设置了 \texttt{AutoFakeBold=1.2} 会导致从 PDF
中复制加粗的字体时变为乱码。

原因解释见刘海洋在\href{https://www.zhihu.com/question/59597144}{知乎}
的回答。

这个问题在多个学校的论文模版中都有反应,解决方案在
\url{https://www.zhihu.com/question/32207411}。 最简单的方式是在
\texttt{xeCJK} 宏包选项中关闭伪粗体,即 \texttt{AutoFakeBold=false}。

此外,在 \texttt{CTeX} 的项目主页 issue
\href{https://github.com/CTeX-org/ctex-kit/issues/353}{xeCJK:
部分汉字的伪粗体在 PDF 文件中无法拷贝或拷贝出异常内容 \#353}
提出,这个问题还与字体选择有关。
Adobe、Fandol、思源黑体/宋体是可以正确复制的,而方正、华文的字体都会出错。

\begin{Shaded}
\begin{Highlighting}[]
\CommentTok{% 选用 Adobe 字体}
\BuiltInTok{\textbackslash{}documentclass}\NormalTok{[fontset=windows]\{}\ExtensionTok{ctexbook}\NormalTok{\}}
\CommentTok{% 或者依次手动设定 Adobe 字体}
\FunctionTok{\textbackslash{}setCJKmainfont}\NormalTok{[AutoFakeBold,ItalicFont=AdobeKaitiStd-Regular]\{AdobeSongStd-Light\}}
\FunctionTok{\textbackslash{}setCJKsansfont}\NormalTok{[AutoFakeBold]\{AdobeHeitiStd-Regular\}}
\FunctionTok{\textbackslash{}setCJKmonofont}\NormalTok{\{AdobeFangsongStd-Regular\}}

\CommentTok{% 选用 Fandol 字体 (缺少生僻字,慎用)}
\BuiltInTok{\textbackslash{}documentclass}\NormalTok{[fontset=fandol]\{}\ExtensionTok{ctexbook}\NormalTok{\}}
\CommentTok{% 或者依次手动设定 Fandol 字体}
\FunctionTok{\textbackslash{}setCJKmainfont}\NormalTok{[AutoFakeBold]\{FandolSong\}}
\FunctionTok{\textbackslash{}setCJKsansfont}\NormalTok{[AutoFakeBold]\{FandolHei\}}
\FunctionTok{\textbackslash{}setCJKmonofont}\NormalTok{\{FandolFang\}}
\end{Highlighting}
\end{Shaded}

上述字体不一定在系统中安装了,采用前需要确认一下本机的字体库。

按照 \texttt{xeCJK} 文档给出的方法,可以用
\texttt{fc-list\ \textgreater{}\ fontlist.txt}
命令把所有字体的信息输入到一个 \texttt{fontlist.txt}
的文本文件中,每一行冒号之前的部分就是字体族名。

特别地,针对中文字体,可以用
\texttt{fc-list\ -f\ "\%\{family\}\textbackslash{}n"\ :lang=zh\ \textgreater{}\ zhfont.txt}
查看。

\subsection{移除 amssymb 包}

在更新了 Tex Live 2020 之后,发现了
\texttt{Command\ \textasciigrave{}\textbackslash{}Bbbk\textquotesingle{}\ already\ defined.}
的报错。

原因是 \texttt{XeLaTeX} 与 \texttt{amsfont} 中有了重复定义。 查阅 AMS
宏包,发现 \texttt{amssymb}
只是提供了部分数学字符的粗体、而且可以被标准的 \texttt{bm} 宏包取代。
故将其移除,仅保留 \texttt{amsmath}。

\begin{itemize}
\item
  未经验证的方案:

  需要把宏包的调用移至 \texttt{ctex} 之前,即可避免出错。
\end{itemize}

\subsection{微分符号}

根据
\href{https://liam.page/2017/05/01/the-correct-way-to-use-differential-operator/}{这篇帖子}
为微积分中的 \(d\) 符号设定了``左侧有间距、直立体''的命令
\texttt{\textbackslash{}dif}。

\section{Geometry
页面尺寸、页眉页脚尺寸}

\subsection{页眉尺寸}

校方模板页眉部分设置有3个参数: * 页边距 25.4mm * 页眉顶端距离 20mm *
正文距离页眉底端 0.7行、即 14bp

如果按照原 \LaTeX 模板的设置,那么页眉\textbf{底缘横线}的位置距离纸张顶部
top-headsep=20mm。

\begin{Shaded}
\begin{Highlighting}[]
\DataTypeTok{top}\OtherTok{=}\FloatTok{25.4}\StringTok{mm,}
\DataTypeTok{headheight}\OtherTok{=}\StringTok{20mm,}
\DataTypeTok{headsep}\OtherTok{=}\FloatTok{5.4}\StringTok{mm,}
\end{Highlighting}
\end{Shaded}

所以跟校方模板的要求相差一个页眉行高。

在默认设置(\texttt{includehead} 和 \texttt{includefoot} 选项都是
false)情况下,
若要求排版效果跟校方模版的样本(而不是跟其描述内容)一致,需要如下设置。

\begin{Shaded}
\begin{Highlighting}[]
\DataTypeTok{top}\OtherTok{=}\FloatTok{30.34}\StringTok{mm,}
\DataTypeTok{headsep}\OtherTok{=}\FloatTok{4.94}\StringTok{mm,}
\DataTypeTok{headheight}\OtherTok{=}\FloatTok{24.81}\StringTok{mm,}
\end{Highlighting}
\end{Shaded}

其中,\texttt{top} 需要加上 0.7行的高度。
\(top = 2.54cm + 0.7*20bp = 25.4mm + 14*(25.4mm/72) = 30.34mm\)

用 \texttt{headsep} 把页眉``顶''上去。 \(headsep = 0.7*20bp = 4.94mm\)。

\texttt{headheight}
计算页眉行行底到页面上边缘距离,需要页眉顶端距离是加上页眉的行高:
\(headheight = 20mm + 10.5bp*1.3 = 24.81mm\)。

\subsection{页脚设置}
校方模板页脚(页码)部分设置有2个参数:
\begin{itemize}[\textbullet]
  \item 页边距 25.4mm
  \item 页脚底端距离 20mm
\end{itemize}

\texttt{\textbackslash{}footskip} 指的是 baseline of last line of text
and baseline of footer 的距离。 按照要求,正好就是页边距减去页码边距
\(25.4mm - 20.0mm = 5.4mm\)。

\subsection{更加优雅的方案}

但是如上处理,发现页眉相关尺寸参数比较``奇怪'',可能回给后续修改者带来困惑。
一个稍微清晰的方案是打开页眉的 \texttt{includehead} 选项。

\begin{Shaded}
\begin{Highlighting}[]
\DataTypeTok{includehead}\OtherTok{=}\KeywordTok{true}\StringTok{,}
\DataTypeTok{top}\OtherTok{=}\FloatTok{20.00}\StringTok{mm,}
\DataTypeTok{headheight}\OtherTok{=}\FloatTok{5.4}\StringTok{mm,}
\DataTypeTok{includefoot}\OtherTok{=}\KeywordTok{false}\StringTok{,}
\DataTypeTok{bottom}\OtherTok{=}\FloatTok{25.4}\StringTok{mm,}
\DataTypeTok{footskip}\OtherTok{=}\FloatTok{5.4}\StringTok{mm,}
\end{Highlighting}
\end{Shaded}

这其实是因为不同的测量点。

\subsection{吐槽}

页码是五号字,即 \(10.5/72*25.4 = 3.70mm\) 左右。
所以说,这样的设置留给页码上边缘和正文下边缘的距离不足 2mm。

校方模板中,这个``紧凑''的边距并不明显,是因为最后一行如果行距不够的情况,会排入下一页;
从而相当于空出了将近一行的距离,正文底线到页底的边距可能超过
\(25.4mm + 20bp = 3cm\);
换句话说,如果全用正文字体多填充几页,就很容易发现这个边距其实挺小的。

页眉部分同理,在 word 中控制``页面上边距 2.54cm,页眉
2.0cm''并且多用正文填充几页,也可以发现,页眉、分割线和正文会挤在一起。

校方的``模板''有的细节经不起推敲,自己取舍吧。

\section{章节标题样式设置}

\subsection{状况描述}

按照手册,\texttt{format=\{\}} 可以统一设置整个标题的格式。

最初,发现 \texttt{\textbackslash{}heiti}
命令对章节编号数字不起作用,仍旧是有衬线字体。 这与校方模版有出入。

在之后的调整中,发现删除
\texttt{nameformat=\textbackslash{}relax},\texttt{numberformat=\textbackslash{}relax}
等语句,会使得章节编号和标题内容字体大小不一。

\subsection{答疑}

参考 \url{https://github.com/CTeX-org/ctex-kit/issues/210} 与
\url{https://github.com/CTeX-org/ctex-kit/issues/422} 的回答。

\begin{quote}
(\texttt{\textbackslash{}ctexset\{\}}) 对标题字体的控制,除了
\texttt{format} 选项之外,还有
\texttt{nameformat}、\texttt{numberformat} 和 \texttt{titleformat}。

这三个选项都在 \texttt{format} 之后起作用。 这里改字号对 chapter
标题无效,是因为它的 nameformat 和 \texttt{titleformat} 默认值都是
\texttt{\textbackslash{}huge\textbackslash{}bfseries},里面的
\texttt{\textbackslash{}huge} 会覆盖你设置的
\texttt{\textbackslash{}zihao\{4\}}。

需要先清空 \texttt{nameformat} 和 \texttt{titleformat} 的设置:
\end{quote}

\begin{Shaded}
\begin{Highlighting}[]
\BuiltInTok{\textbackslash{}documentclass}\NormalTok{[zihao=-4]\{}\ExtensionTok{ctexbook}\NormalTok{\}}
\FunctionTok{\textbackslash{}ctexset}\NormalTok{\{}
\NormalTok{    chapter/format      = }\FunctionTok{\textbackslash{}zihao}\NormalTok{\{4\}}\FunctionTok{\textbackslash{}bfseries}\NormalTok{,}
\NormalTok{    chapter/nameformat  = \{\},}
\NormalTok{    chapter/titleformat = \{\},}
\NormalTok{    section/format      = }\FunctionTok{\textbackslash{}zihao}\NormalTok{\{4\},}
\NormalTok{\}}
\end{Highlighting}
\end{Shaded}

\begin{quote}
两种不同排版方案的实现方法不统一,优先级的问题。

\texttt{scheme=chinese} 时,章节标题统一由 \texttt{format} 给出,而
\texttt{nameformat} 与 \texttt{titleformat} 为空。

\texttt{scheme=plain} 时,章节标题分别由 \texttt{nameformat} 与
\texttt{titleformat} 给出,而 \texttt{format} 为空。这是因为在 LaTeX2e
标准文档类里,「Chapter XX」是用 \texttt{\textbackslash{}huge}
字号,而标题内容则是用 \texttt{\textbackslash{}Huge} 字号。

简单解决方法:\texttt{scheme=plain} 时,根据需要清空 \texttt{nameformat}
或 \texttt{titleformat。}
\end{quote}

\begin{Shaded}
\begin{Highlighting}[]
\BuiltInTok{\textbackslash{}documentclass}\NormalTok{[scheme=plain]\{}\ExtensionTok{ctexbook}\NormalTok{\}}
\FunctionTok{\textbackslash{}ctexset}\NormalTok{\{}
\NormalTok{  chapter=\{}
\NormalTok{    format=}\FunctionTok{\textbackslash{}small}\NormalTok{,}
\NormalTok{    titleformat=\{\},}
\NormalTok{  \}}
\NormalTok{\}}
\end{Highlighting}
\end{Shaded}

此外,在 \url{https://github.com/CTeX-org/ctex-kit/issues/422}
的后续回答中,提到中英文字体不同。

\begin{quote}
mohuangrui: 请教一下,对于章节标题,是否可能通过 \texttt{format} 实现
中文 调用 \texttt{\textbackslash{}sffamily} 而 英文和数字 调用
\texttt{\textbackslash{}rmfamily}?

RuixiZhang42:可以用 \texttt{\textbackslash{}heiti}
这样的命令,因为只作用于 CJK
文字上。不过这样西文跟中文不太协调,建议谨慎使用

muzimuzhi:字体切换功能,是利用了 xetex 引擎的 char class
机制。粗暴理解,把所有字符按中西文分类,遇到「中-西」边界就切换到西文字体,遇到「西-中」边界就切换到中文字体。{[}1,
Sec. 3{]} 里有一个简单的例子。

muzimuzhi:zepinglee 提供的建议是正确且有效的。更一般的方案是,通过
\texttt{fontspec} 和 \texttt{xeCJK} 的
\texttt{\textbackslash{}new{[}CJK{]}fontfamily}
命令,分别定义中西字体切换命令,然后一起使用。
\end{quote}

\begin{Shaded}
\begin{Highlighting}[]
\CommentTok{% def}
\FunctionTok{\textbackslash{}newfontfamily\textbackslash{}useSomeEnFont}\NormalTok{\{<font name 1>\}}
\FunctionTok{\textbackslash{}newCJKfontfamily\textbackslash{}useSomeEnFont}\NormalTok{\{<font name 2>\}}
\CommentTok{% use}
\FunctionTok{\textbackslash{}normalfont\textbackslash{}useSomeEnFont\textbackslash{}useSomeEnFont}
\end{Highlighting}
\end{Shaded}

\subsection{处理}

在 \texttt{ctexset} 中,章节标题在不同 schema
下已经有默认值,如果不取消则会继承,所以如果要定制章节标题样式,需要记得取消默认设置。
* 通常,\texttt{\textbackslash{}section}
以下的默认设置均为空(\texttt{=\{\}}),比较省心。

对于,ctex 的中文字体命令调用了的还是 xeCJK,其\textbf{只}对 CJK
中日韩文字有效果,而对英文和数字不起效果。所以采用 \texttt{ctex} 提供的
\texttt{\textbackslash{}songti},\texttt{\textbackslash{}heiti}
等命令均不会影响英文字母和数字的字体。

如果必要调整字体,可以用宏包 \texttt{\{fontspec\}} 处理。

\begin{enumerate}
\def\labelenumi{\arabic{enumi}.}
\item
  最简单的方法是,直接设置全文的 \texttt{\textbackslash{}sffamily}
  字体为黑体,即添加 \texttt{\textbackslash{}setsansfont\{SimHei\}}
  一句。

  \begin{itemize}
  \item
    然而,黑体作为面向中文的字体,显示字母和数字并不美观。
  \item
    尤其明显地,章节编号的分隔符(.号)变为全角宽度。
  \end{itemize}
\item
  如果放弃``黑体'',可以选择``雅黑''等针对中英文混排优化过的无衬线字体。
\item
  然而,即使在 MS Office 中,常用的处理方案也是用有衬线字体。

  \begin{itemize}
  \item
    这也是更加美观的选择。
  \end{itemize}
\end{enumerate}

最终,删除了原 \texttt{romantitle} 布尔型选项,仿照 \texttt{degreetype}
新增了名为 \texttt{titlenum} 的三选一方案(意为 \emph{title}
\emph{num}ber)。

\begin{itemize}
\item
  默认值 \texttt{rmtitlenum} 即采用 Times New Roman
  衬线字体(\texttt{\textbackslash{}rmfamily});
\item
  可以选择 \texttt{sftitlenum}
  即无衬线的半角字体(\texttt{\textbackslash{}sffamily});
\item
  可以选择 \texttt{heititlenum}
  校方模版要求的黑体(\texttt{SimHei},不是
  \texttt{\textbackslash{}heiti}),
\end{itemize}

\begin{quote}
说实话,个人认为黑体数字编号确实不如衬线字体好看;
而且分割号(实心圆点)是全角字符,也有点别扭。
\end{quote}

\subsection{节外生枝}

有时候标题 \texttt{第1章} 或 \texttt{(1)} 中数字并不居中。

如果前后字体不一样,例如 \texttt{nameformat} 与 \texttt{numberformat}
分别选用无衬线和衬线字体,或者进行中英文混排时, 好像会触发
\texttt{xeCJK} 的 \texttt{CJKecglue} 设置,使其自动追加空格。

所以,设置 \texttt{numberformat=\{\}},统一在 \texttt{nameformat}
中设置格式。 可以在
\texttt{ctexset=\{chapter/name=\{第\textbackslash{},,章\}\}}
中的``第''字之后手动加一个六分之一小空格 \texttt{\textbackslash{},}
以规避 \texttt{xeCJK} 自作主张的小聪明。

\subsection{强迫症的补完}

\texttt{ctex} 提供的章节标题命令最小是
\texttt{\textbackslash{}subparagraph}、即5级标题,形如``x.x.x.x.x.x''共6个数字。

出于强迫症考虑,把所有的样式都统一定义了。

\begin{enumerate}
\def\labelenumi{\arabic{enumi}.}
\item
  开启 5 级标题编号:

  \begin{enumerate}
  \def\labelenumii{\arabic{enumii}.}
  \item
    必须设置章节编号深度
    \texttt{\textbackslash{}setcounter\{secnumdepth\}=5}。
  \item
    可以在 \texttt{\textbackslash{}LoadClass} 时,为 \texttt{ctex}
    添加上 \texttt{sub4section}
    选项(没有true/false值)。但是默认效果不好,还是需要详细设置。
  \end{enumerate}
\item
  样式定义,方法同 \texttt{section} 等,不再赘述。

  \begin{itemize}
  \item
    paragraph (4级)形如 \texttt{(1)\ 段落名}。

    \begin{itemize}
    \item
      样式延续各级 section:黑体、顶格。
    \item
      编号不含上级,仅为全角括号与一个阿拉伯数字。
    \end{itemize}
  \item
    subparagraph (5级)形如 \texttt{①\ 子段落名}。

    \begin{itemize}
    \item
      样式同正文,宋体,缩进2字符。
    \item
      编号是圆圈数字。
    \end{itemize}
  \end{itemize}
\end{enumerate}

以前就觉得 \texttt{paragraph} 和 \texttt{subparagraph}
的等级太详细了、且样式上跟列表环境有相似。
曾用字体(黑体、楷体)以示与正文列表的宋体字体区分,但仍然不尽人意。

现在,编号方式向老板编写的教材中的编号习惯看齐,应该没有问题了吧?

\begin{quote}
区别在于 paragraph 等级我仍然采用了黑体字体,老板用的是正文宋体字体了。
\end{quote}

\section{分章节编译问题}

\textbf{前排提示}:
这项修改纯属本人对于文件夹的某种奇怪强迫症而已,望同学们不要模仿------这项功能毫无意义。

分章节独立编译常用宏包是 \texttt{\{subfiles\}} 和
\texttt{\{standalone\}} 两个。
前者更加简单;后者提供了更多的控制选项,代价是更复杂一些。

宏包 \texttt{\{subfiles\}} 文档
\url{http://ctan.math.washington.edu/tex-archive/macros/latex/contrib/subfiles/subfiles.pdf}。

\subsection{状况描述}

如果把章节 .tex
文档放到另一个文件夹中,则宏包、参考文献和图片的路径相对位置是不同的。
会编译报错,比如找不到文件。

\subsection{答疑}

可以把 main.tex 文件也放到一个子文件夹内,则
\texttt{\textbackslash{}documentclass\{\}} 和
\texttt{\textbackslash{}usepackage\{\}} 的都可以采用 \texttt{../xxx}
的相对路径。

\subsection{处理}

\subsubsection{项目目录结构处理}

调整过后的文档结构示意如下。

\begin{verbatim}
..  % 主文件夹 Master Folder
|-- appendix\ % 附录
    |-- appendix.tex % 附录章节,可以在一个.tex文件中分\chapter,也可以每章一个.tex文件。
|-- body\ % 正文
    |-- chapter1.tex % 单独章节
    |-- chapter2.tex
|-- code\ % 存放代码,用于algorithm和listing环境引用。
|-- figure\ % 各类图片
    |-- fig1,2,...[png|pdf|eps|...]
|-- pages\ % 格式化的页面
    |-- abstract.tex % 摘要,以{environment}形式
    |-- acknowledge.tex % 致谢,以{environment}形式
    |-- cover.tex % 封面信息,填表
    |-- denotation.tex % 符号对照表,以{environment}形式(description列表)
    |-- resume.tex % 个人简历,以{environment}形式
|-- main\
    |-- Thesis.tex % 组装成完整的论文
|-- ref\
    |-- references.bib % 参考文献
|-- THEME\ % 用于存放模板样式文件
    |-- tongjithesis.cls % class文件
    |-- tongjithesis.sty % package文件
    |-- tongjithesis.cfg % 配置 configuration 文件
\end{verbatim}

于是,主文件和单独章节文件可以引用相同的相对路径,就可以分章节编译的目的。

\subsubsection{在主文件 main.tex
中的调整}

各种路径都要先用\texttt{../}回到上一层、然后再指定平级目录、最后指定文件。
虽然这种操作不是很常见,但是相比正常路径,也就多了\texttt{../}这一步,算是可以接受。

\begin{Shaded}
\begin{Highlighting}[]
\BuiltInTok{\textbackslash{}documentclass}\NormalTok{[bibtype=numeric]\{}\ExtensionTok{../THEME/tongjithesis}\NormalTok{\} }\CommentTok{% 1. 文档类型}
\BuiltInTok{\textbackslash{}usepackage}\NormalTok{\{}\ExtensionTok{../THEME/tongjithesis}\NormalTok{\} }\CommentTok{% 2. 宏包}
\KeywordTok{\textbackslash{}begin}\NormalTok{\{}\ExtensionTok{document}\NormalTok{\}}
\NormalTok{  ...}
  \FunctionTok{\textbackslash{}subfile}\NormalTok{\{../body/chapter.tex\} }\CommentTok{% 3. 章节文件}
\KeywordTok{\textbackslash{}end}\NormalTok{\{}\ExtensionTok{document}\NormalTok{\}}
\end{Highlighting}
\end{Shaded}

\subsubsection{在子文件 chapter.tex
中的调整}

子文件中按照 subfile 的要求即可,不需要
\textbackslash usepackage,然后用\{document\}环境包围正文代码。

\begin{Shaded}
\begin{Highlighting}[]
\BuiltInTok{\textbackslash{}documentclass}\NormalTok{[../main/BMS_BIM.tex]\{}\ExtensionTok{subfiles}\NormalTok{\} }\CommentTok{% 1. 文档类型选项}
\KeywordTok{\textbackslash{}begin}\NormalTok{\{}\ExtensionTok{document}\NormalTok{\}}
\NormalTok{  ...}
\KeywordTok{\textbackslash{}end}\NormalTok{\{}\ExtensionTok{document}\NormalTok{\}}
\end{Highlighting}
\end{Shaded}

\subsubsection{在模板文件中的调整}

为了避免如下的报错提醒,

\begin{verbatim}
You have requested document class `../THEME/tongjithesis',
           but the document class provides `tongjithesis'.
\end{verbatim}

可以在模板文件中修改 \texttt{.cls} 和 \texttt{.sty} 引用名。

\begin{Shaded}
\begin{Highlighting}[]
\FunctionTok{\textbackslash{}ProvidesClass}\NormalTok{\{../THEME/tongjithesis\}}
\FunctionTok{\textbackslash{}ProvidesPackage}\NormalTok{\{../THEME/tongjithesis\}}
\end{Highlighting}
\end{Shaded}

\subsection{处理(改良)}

手动安装 \texttt{.cls} 和 \texttt{.sty} 文件到系统,即可避免这些问题。

方法参考:\href{https://zhuanlan.zhihu.com/p/113124407}{手动安装 sty 和
cls 文件}

\subsubsection{主要步骤}

\begin{enumerate}
\def\labelenumi{\arabic{enumi}.}
\item
  打开终端

  \begin{itemize}
  \item
    Windows 上的 cmd.exe 或 PowerShell.exe,
  \item
    macOS 上的 Terminal.app 。
  \end{itemize}
\item
  输入
  \texttt{kpsewhich\ -\/-var-value=TEXMFHOME},按回车,得到一个路径,比如说
  \texttt{X:/xxx/texmf} 。

  \begin{enumerate}
  \def\labelenumii{\arabic{enumii}.}
  \item
    打开这个路径(文件夹)。
  \item
    如果路径中的某一层或几层文件夹不存在,就创建它们。
  \item
    最后会位于 \texttt{texmf/} 文件夹。
  \end{enumerate}
\item
  在 \texttt{texmf/} 中继续创建两层文件夹 \texttt{tex/latex/},
  把现在的路径为 \texttt{X:/xxx/texmf/tex/latex/} 。
\item
  在 \texttt{latex/} 文件夹中创建一个文件夹,名称任选(例如为
  \texttt{my-pkg/})。 名称其实不关键,只是用于管理文件罢了。
\item
  把 \texttt{xxx.cls} 和 \texttt{xxx.sty} 文件放进文件夹
  \texttt{my-pkg/}。 最终完整路径形如:

  \begin{itemize}
  \item
    \texttt{X:/xx/texmf/tex/latex/my-pkg/xxx.cls}
  \item
    \texttt{X:/xx/texmf/tex/latex/my-pkg/xxx.sty}
  \end{itemize}
\end{enumerate}

结束。

现在,本机任何位置的 \texttt{.tex}
文件都可以取用这些类文件(\texttt{xxx.cls})和宏包(\texttt{xxx.sty})。

\subsubsection{讨论}

\begin{itemize}
\item
  命令 \texttt{kpsewhich} 是 \TeX Live 命令,用法参看
  \texttt{texlive-en.pdf};如果是其他发行版,需要另行找资料。
\item
  「安装」在 \texttt{TEXMFHOME}
  中的文件,只有(登陆操作系统的)当前用户可以使用。

  \begin{itemize}
  \item
    如果希望为所有用户「安装」,可以把第二步中的 TEXMFHOME 替换为
    TEXMFLOCAL。
  \item
    此时,还需要第七步:以管理员权限在终端执行 \texttt{texhash} 命令。
  \end{itemize}
\item
  变量 \texttt{TEXMFHOME}
  的值(第二步中得到的路径)可以修改,还可以包括多个路径。
\item
  变量 \texttt{TEXMFHOME}
  储存的每一个路径,都要满足特定的目录结构要求,称为 TDS (TeX Directory
  Structure)。有时也叫做 texmf 树。在步骤介绍中,

  \begin{itemize}
  \item
    \texttt{tex/latex/} 两层目录是 TDS 强制的,表示【这是 latex
    格式的宏包文件】。
  \item
    \texttt{my-pkg/} 一层,体现的是包名(package
    name),可以任取。有多个包时,可以在 \texttt{tex/latex/}
    下建立多个文件夹。
  \end{itemize}
\item
  可以用符号链接的方式把宏文件(包括但不限于 sty, cls, tex, def 等)加到
  TDS 的特定子目录中。符号链接能进一步提高宏文件的可维护性。
\end{itemize}

\section{宏包 gb7714 报错}

\subsection{状况描述}

报错提醒如下。

\begin{verbatim}
Package xkeyval Error: `gbnamefmt` undefined in families `blx@opt@pre`.
% 其中`gbnamefmt`也可会是`urldate`等其他信息
...
\blx@processoptions
\end{verbatim}

\subsection{答疑}

需要手动更新适用于中文文献的宏包,本次即 \texttt{biblatex-gb7714-2015}。

根据 \href{https://github.com/hushidong/biblatex-gb7714-2015}{宏包
gb7714} 的说明文档:

\begin{quote}
最简单的方法是从本项目源码中下载 gb7714-2015.bbx, gb7714-2015ay.bbx,
gb7714-2015.cbx, gb7714-2015ay.cbx 四个文件
放到你要编译的主文档所在目录,如果需要使用gbk编码,则还需复制
gb7714-2015-gbk.def 文件。
对于已经安装的用户需要更新到最新版,则可以下载这些文件替换系统已经安装的文件。
\end{quote}

\subsection{处理}

下载四个文件,并替换
\texttt{\$TexLive\$/texmf-dist/tex/latex.biblatex-gb7714-2015/}
文件夹内原有内容。 * gb7714-2015.bbx * gb7714-2015ay.bbx *
gb7714-2015.cbx * gb7714-2015ay.cbx

原README也指出了这个问题。 此外,如下论坛帖也有详细说明。
https://blog.genkun.me/post/xjtu-undergraduate-thesis/

\section{Biber 版本错误}

\subsection{状况描述}

报错提醒如下。

\begin{verbatim}
ERROR - Error: Found biblatex control file version 3.4, expected version 3.5.
This means that your biber (2.12) and biblatex (3.11) versions are incompatible.
See compat matrix in biblatex or biber PDF documentation.
\end{verbatim}

本项目编译时(2019年3月),TexLive 提供的的最新版本 biber 2.1 与 biblatex
3.12 不能兼容。

其官方论坛也承认了这个bug,且唯一办法就是 (downgrade
biber){[}https://bugs.archlinux.org/task/60844{]}。

\subsection{答疑}

根据biblatex.pdf的Table.1说明,下载了biber2.1并覆盖到
TexLive安装\textbackslash bin\textbackslash win32下。
(GT的电脑是 c:\textbackslash Program Files\textbackslash texlive\textbackslash 2018\textbackslash bin\textbackslash win32 )

\begin{quote}
可在cmd中用 \texttt{where\ biber} 查询biber的位置,也可以搜索biber.exe
定位。
\end{quote}

\subsubsection{biber历史版本下载}

https://sourceforge.net/projects/biblatex-biber/files/biblatex-biber/2.11/

\section{公式索引}

\LaTeX 原生没有``公式索引''这样的命令,所以是从清华的模板中学到的命令。

因为公式数量庞大,所以罗列所有公式通常不是什么好的决定。
所以清华方法只会把指定公式列入索引。

他们定义了一个新命令 \texttt{\textbackslash{}equcaption\{\}}
用来标定重要公式,然后 \texttt{\textbackslash{}listofequations}
只列出这些被标记的公式。 通常,建议用 \texttt{amsmath} 宏包提供的
\texttt{\textbackslash{}tag} 命令一起使用。

\section{电子书签}

目录和电子书签不同,许多章节不会加入目录,但是为了方便通常会加入电子书签,以便跳转。

\begin{itemize}
\item
  【目录】 是会被打印出来的目录,包括 \texttt{mainmatter} 和
  \texttt{backmatter}。 在文档中由
  \texttt{\textbackslash{}tableofcontents} 命令生成,由
  \texttt{titletoc} 宏包控制样式。
\item
  【PDF电子书签】 是用 PDF
  阅读器打开的电子版文件才出现的目录,可以更全面。
\end{itemize}

\subsection{添加目录项}

通常的章节命令会自动加入目录,不赘述。

为了把不编号章节同时加入目录和 PDF
电子书签,需要在章节标题命令\textbf{之后}手动添加
\texttt{\textbackslash{}addcontentsline\{ext\}\{depth\}\{title\}} 命令。

\begin{itemize}
\item
  第一个参数 \texttt{ext} 表示加入目录或者索引。 可选为:

  \begin{itemize}
  \item
    目录 \texttt{toc} (table of contents),
  \item
    图片索引 \texttt{lof} (list of figures),
  \item
    表格索引 \texttt{lot} (list of tables)。
  \end{itemize}
\item
  第二个参数 \texttt{depth} 表示添加后的等级(层次、或称深度),比如
  \texttt{part}、\texttt{chapter}、\texttt{section}等;
\item
  第三个参数 \texttt{title} 表示它在目录 / 索引中的展示的标题名字。
\end{itemize}

\subsection{制作电子书签}

原模版的 PDF 书签采用 \texttt{hyperref} 宏包的
\texttt{\textbackslash{}pdfbookmark{[}level{]}\{text\}\{name\}} 命令,为
PDF 文档生成与文档目录部分内容和结构相同的书签。

使用方法可以参考 \url{https://zhuanlan.zhihu.com/p/59655379}
一文,需要控制书签所在的层次、书签的标题、书签的跳转目标三个参数。
同样,这个命令与必须跟随在章节标题命令\textbf{之后}。

然而,直接手动添加的书签,经常跳转目标不准确。 如果要正确跳转,先
\texttt{\textbackslash{}phantomsection} 产生一个虚假的章节,之后添加
\texttt{\textbackslash{}pdfbookmark}。
而且,这种方式需要编译两次才能出现电子书签。

\subsection{自定义的命令}

正如开始所说,章节在文章目录 \texttt{TOC} 和 PDF
电子书签中的应当分离控制。
比如封面、目录本身、授权页面等不出现在目录,但最好出现在电子书签。

原模版借鉴了清华模版,采用了带有的四个参数自定义\textbf{章}标题命令
\texttt{\textbackslash{}tongji@chapter\{s\ o\ m\ o\}},从而更加自由灵活地控制目录和书签的展示行为。

\begin{itemize}
\item
  第一个是星号参数
  \texttt{s},用于提醒此新命令与原生命令的必须区别,不会实际更改目录或者电子书签的内容或格式。
\item
  第二个是可选参数 \texttt{o},表示
  \texttt{{[}tocline{]}},是出现在目录中的条目。

  \begin{itemize}
  \item
    如果有、且为空 \texttt{{[}{]}},则此章节不出现在目录中;
  \item
    如果有、且不为空 \texttt{{[}mark{]}},则在电子书签中显示为
    \texttt{mark};
  \item
    如果没有,表示在电子书签中显示 \texttt{\{title\}}。
  \end{itemize}
\item
  第三个是必需参数 \texttt{m},表示 \texttt{\{title\}},即章标题
  \texttt{\textbackslash{}chapter\{\}} 括号中的内容。
\item
  第四个是可选参数 \texttt{o},表示
  \texttt{{[}header{]}},即页眉出现的标题。

  \begin{itemize}
  \item
    如果有,则成为页眉;
  \item
    如果没有,则取 \texttt{\{title\}}作为页眉。
  \end{itemize}
\end{itemize}

总结来说, 第一个参数要求必须以 \texttt{\textbackslash{}tongji@chapter*}
的形式生成章节并进入电子书签; 第三个参数是必需的章标题名字;
第二、四个参数是其它相关位置显示的名字,且区分了无参数、有空参数、有非空参数三种情况。

几个用法举例。

\begin{enumerate}
\def\labelenumi{\arabic{enumi}.}
\item
  中文摘要

  \texttt{\textbackslash{}tongji@chapter*{[}{]}\{\textbackslash{}cabstractname\}{[}\textbackslash{}wuhao\textbackslash{}songti\textbackslash{}tongji@schoolname\textasciitilde{}\textbackslash{}tongji@capply\textasciitilde{}\textbackslash{}cabstractname{]}}

  \begin{longtable}[]{@{}cll@{}}
  \toprule
  参数 & 内容 & 解释\tabularnewline
  \midrule
  \endhead
  s & \texttt{*} & 确认带星号\tabularnewline
  o & \texttt{{[}{]}} & 有且为空,不出现在目录\tabularnewline
  m & \texttt{\{\textbackslash{}cabstractname\}} &
  ``摘要''二字\tabularnewline
  o &
  \texttt{{[}\textbackslash{}wuhao\ ...\ \textbackslash{}cabstractname{]}}
  & 摘要的页眉\tabularnewline
  \bottomrule
  \end{longtable}

  英文摘要、目录同理。
\item
  致谢

  \texttt{\textbackslash{}tongji@chapter*{[}\textbackslash{}tongji@ackname{]}\{\textbackslash{}tongji@ackname\}{[}\textbackslash{}wuhao\textbackslash{}songti\textbackslash{}tongji@schoolname\textasciitilde{}\textbackslash{}tongji@capply\textasciitilde{}\textbackslash{}tongji@ackname{]}}

  \begin{longtable}[]{@{}cll@{}}
  \toprule
  参数 & 内容 & 解释\tabularnewline
  \midrule
  \endhead
  s & \texttt{*} & 确认带星号\tabularnewline
  o & \texttt{{[}\textbackslash{}tongji@ackname{]}} &
  有且不为空,出现在目录中显示 \texttt{\textbackslash{}tongji@ackname}
  对应的``致谢''二字\tabularnewline
  m & \texttt{\{\textbackslash{}tongji@ackname\}} &
  ``致谢''二字\tabularnewline
  o &
  \texttt{{[}\textbackslash{}wuhao\ ...\ \textbackslash{}tongji@ackname{]}}
  & 致谢的页眉\tabularnewline
  \bottomrule
  \end{longtable}
\item
  不带星号的 \texttt{\textbackslash{}listof}

  \texttt{\textbackslash{}tongji@chapter*\{\textbackslash{}csname\ list\#2name\textbackslash{}endcsname\}{[}\textbackslash{}wuhao\textbackslash{}songti\textbackslash{}tongji@schoolname\textasciitilde{}\textbackslash{}tongji@capply\textasciitilde{}\textbackslash{}csname\ list\#2name\textbackslash{}endcsname{]}}

  \begin{longtable}[]{@{}cll@{}}
  \toprule
  参数 & 内容 & 解释\tabularnewline
  \midrule
  \endhead
  s & & 确认带星号\tabularnewline
  o & & 没有,在目录显示 \texttt{m} 参数的文字\tabularnewline
  m &
  \texttt{\{\textbackslash{}csname\ list\#2name\textbackslash{}endcsname\}}
  & 对应的``索引''名字\tabularnewline
  o &
  \texttt{{[}\textbackslash{}wuhao\ ...\ \textbackslash{}endcsname{]}} &
  索引的页眉\tabularnewline
  \bottomrule
  \end{longtable}

  正文的章节都是这种类型。
\end{enumerate}

\subsection{优化电子书签}

\url{https://zhuanlan.zhihu.com/p/66434387} 指出,在 \texttt{hyperref}
宏包的用户手册中,针对 PDF 书签的功能,也推荐用户使用 \texttt{bookmark}
宏包(见 \texttt{hyperref} 手册 4.1.1 节末尾),而不再推荐用
\texttt{hyperref} 了。

\begin{itemize}
\item
  如果是已经 \texttt{\textbackslash{}addcontentsline} 而加入了
  \texttt{toc} 的项目,会自动进入电子书签;
\item
  如果是不在 \texttt{toc} 中的章节,在其章节命令
  \texttt{\textbackslash{}chapter*\{\}} 之后用
  \texttt{\textbackslash{}bookmark{[}dest=\textbackslash{}HyperLocalCurrentHref,\ level={]}\{title\}}
  设定其在电子书签中的级别和标题内容即可。
\end{itemize}

从而,在
\texttt{\textbackslash{}NewDocumentCommand\textbackslash{}tongji@chapter}
中不再需要 \texttt{\textbackslash{}phantomsection} 了。

不过,如果要为封面、授权说明和原创性声明添加电子书签,还是需要
\texttt{\textbackslash{}phantomsection},因为它们中原本不含有章节命令,无法被定位。

值得注意,这并不意味着可以不用加载 \texttt{hyperref} 宏包。
\texttt{hyperref} 宏包提供了全面的超链接功能,\texttt{bookmark}
应该理解为其一个扩展。 比如,默认的超链接都有一个红色矩形框包裹,需要在
\texttt{hyperref} 的设置中 \texttt{colorlinks=false} 以禁用;此外,PDF
文件的小写罗马字母、大写罗马字母和阿拉伯数字页码仍然需要编译两次才能正确显示。

\section{To be continued
遗留问题}

\subsection{用ltxdoc包生成dtx文件}

这个需要对整个模板工程有全局的统筹,工作量太大,估计必要分工协作才行。
只能留给后面有心的同学吧。

\subsection{封面样式}

\begin{itemize}
\item
  顶部标签影响竖向布局

  中文封面中,如果最上方添加了``打印时删除''``电子版''或者``保密''的标签字样时候,原本的
  \texttt{\textbackslash{}parbox} 的 24bp 的高度设置会被改变。
  与不添加这些标签字样相比,相差了 12bp 的高度。

  尝试更改字体的行间距,无效。

  原理仍有待探究。

  现在的解决方法是把中文封面的段落盒子高度增加为
  \texttt{\textbackslash{}parbox{[}t{]}{[}36bp{]}{[}t{]}\{\textbackslash{}textwidth\}\{...\}};
  而英文封面中没有这些元素,用 \texttt{\textbackslash{}vspace*\{24bp\}}
  跳过。 则两种封面的其他元素可以顺利对齐。
\item
  封面表格 \texttt{Underfull\ \textbackslash{}hbox} 提醒

  中文封面的申请人信息表格,第一、二行``姓名''、``学号''因为字数太少,会有
  \texttt{Underfull\ \textbackslash{}hbox} 的提醒。 所以在 \texttt{.cfg}
  文件中两个汉字之间手动加入了 \texttt{\textbackslash{}hfill}
  命令用于扩充,并取消了这两行原本的
  \texttt{\textbackslash{}tongji@put@title\{\}} 命令。

  \begin{itemize}
  \item
    当然,如果是有副导师的情况,则``所在院系''``学科门类''等四字列仍然会有
    \texttt{Underfull} 的提醒,所以不必担心。
  \end{itemize}
\end{itemize}

\subsection{分章节编译产生的问题}

分章节的编译方法好像没有产生太大的好处,而且在使用过程中还遇到了更多的问题。

\begin{itemize}
\item
  必须 magic comment

  需要在章文件开头添加 \texttt{\%\ !TEX\ program\ =\ xelatex} 才能使
  \texttt{latexmk} 顺利编译单独章节,否则会调用默认的 \texttt{pdflatex}
  编译。

  可以在章文件开头添加 \texttt{\%\ !TeX\ root\ =\ ../Main/final.tex}
  使得 \texttt{latexmk} 自动进行根目录的编译。
\item
  表格中字号错误

  单章编译时,表格字号不是在 \texttt{.cls}
  中定义的字号(五号),而是成为正文字号;
  然而,如果回到整合的文件编译,则不会出现这种状况,表格文字被顺利的缩小为五号。
\end{itemize}

TODO:对于分章节编译,需要盖棺论定。
等到论文写完,对比过程中分章节编译的优劣势。 因为 \texttt{latexmk}
好像可以加速编译,所以如果单章节的编译与总编译的时间相差不多,那么分章节的优势可能就仅在于``方便分章节检查''这样的了------然而有同步、有电子书签的情况下,也没那么大优势。

\section{原版说明文件}

\begin{itemize}
\item
  \href{https://github.com/marquistj13/TongjiThesis/blob/master/README.md}{README.md}
\item
  \href{https://github.com/marquistj13/TongjiThesis/blob/master/changes.md}{CHANGES.LOG}
\end{itemize}

\end{document}
