% !TEX program = xelatex
\documentclass[degree=doctor,bibtype=numeric,pifootnote=true,]{tongjithesis}
% 选项列表(*代表默认选项)
%   degree=[master*|doctor],% 必选:硕士* 或 博士
%   bibtype=[numeric*|authoryear],% 可选:数字式上标* 或 作者-年份
%   degreetype=[academic*|profession|equaleducation],% 可选:学术型* 或 专业型 或 同等学力
%   titlenum=[rmtitlenum*|sftitlenum|heititlenum],% 可选:章节编号字体:有衬线* 或 无衬线 或 黑体
%   electronic=[true|false*],% 可选:封面左上角“打印时删除”显示与否
%   draft=[true|false*],% 可选:是否减少空白页以求页面紧凑,会在封面打印“电子版”字样作为提示。
%   secret=[true|false*],% 可选:是否保密
%   raggedbottom=[true*|false],% 可选:优化页面布置
%   pifootnote=[true*|false],% 可选:页面尾注是否是圆圈序号

% 所有其它可能用到的包都统一放到这里了,可以根据自己的实际添加或者删除。
\usepackage{tongjithesis}

% 填充废话专用宏包
\usepackage{zhlipsum}

% 参考文献更新使用biblatex包, 使用gb7714-2015标准, 具体参数设置可在cls文件中搜索biblatex进行了解
\addbibresource{../ref/refs.bib}% 加入bib文件
\addbibresource{../ref/custom.bib}% 多个文献来源

% 定义所有的eps文件在 figures 子目录下
\graphicspath{{../figures/}}

\begin{document}

\makecover*{../pages/cover}% 中英文封面,带*则无书脊
\makeauthorizationpage%[../pages/scanauth.pdf]% 授权书
\makedeclarepage%[../pages/scandecl.pdf]% 原创声明

\frontmatter%%% 论文前段
\input{../pages/abstract}% 中英文摘要
\tableofcontents% 目录
\input{../pages/denotation}% 符号对照表

\mainmatter%%% 正文
\subfile{../body/chap01}
\subfile{../body/chap02}
\subfile{../body/chap03}

\backmatter%%% 其它部分
\input{../pages/ack}% 致谢
%% 索引,星号影响是否出现在目录页中
\listoffigures%*% 插图索引
\listoftables%*% 表格索引
\lisfoflistings% 代码索引
\listofequations*% 公式索引
\printTJbibliography% 参考文献
\begin{appendix}% 附录
  \subfile{../pages/appendix.tex}
\end{appendix}
\input{../pages/resume}% 个人简历

\end{document}
