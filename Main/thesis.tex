% !TEX program = xelatex
\documentclass[degree=master,bibtype=numeric,footnotenum=pifont,]{tongjithesis}
% 选项列表(*代表默认选项)
%   degree=[master*|doctor],% 必选:硕士* 或 博士
%   bibtype=[numeric*|authoryear],% 可选:数字式上标* 或 作者-年份
%   degreetype=[academic*|profession|equaleducation],% 可选:学术型* 或 专业型 或 同等学力
%   titlenum=[rmtitlenum*|sftitlenum|heititlenum],% 可选:章节编号字体:有衬线* 或 无衬线 或 黑体
%   electronic=[true|false*],% 可选:封面左上角“打印时删除”显示与否,且减少空白页,控制超链接样式
%   secret=[true|false*],% 可选:是否保密
%   raggedbottom=[true*|false],% 可选:优化页面布置
%   footnotenum=[pifont*|circle|normal],% 可选:页面脚注编号类型:Pifont字体* 或 圆圈 或 普通无修饰

% 所有其它可能用到的包都统一放到这里了,可以根据自己的实际添加或者删除。
\usepackage{tongjithesis}

% 填充废话专用宏包
\usepackage{zhlipsum}

% 参考文献更新使用biblatex包, 使用gb7714-2015标准, 具体参数设置可在cls文件中搜索biblatex进行了解
\addbibresource{../ref/refs.bib}% 加入bib文件
\addbibresource{../ref/custom.bib}% 多个文献来源

% 定义所有的eps文件在 figures 子目录下
\graphicspath{{../figures/}}

\begin{document}
%%% 生成特殊页面
\makecover{../pages/cover}% 中英文封面,带*则无书脊
\makeauthorizationpage%[../pages/scanauth.pdf]% 授权书
\makedeclarepage%[../pages/scandecl.pdf]% 原创声明
%%% 论文前段
\frontmatter%
% 定义中英文摘要和关键字
\begin{cabstract}
  论文的摘要是对论文研究内容和成果的高度概括。摘要应对论文所研究的问题及其研究目
  的进行描述,对研究方法和过程进行简单介绍,对研究成果和所得结论进行概括。摘要应
  具有独立性和自明性,其内容应包含与论文全文同等量的主要信息。使读者即使不阅读全
  文,通过摘要就能了解论文的总体内容和主要成果。

  论文摘要的书写应力求精确、简明。切忌写成对论文书写内容进行提要的形式,尤其要避
  免“第 1 章……;第 2 章……;……”这种或类似的陈述方式。

  本文介绍同济大学论文模板 \tongjithesis{} 的使用方法。本模板符合学校的硕士、
  博士论文格式要求。

  本文的创新点主要有:
  \begin{itemize}
    \item 用例子来解释模板的使用方法;
    \item 用废话来填充无关紧要的部分;
    \item 一边学习摸索一边编写新代码。
  \end{itemize}

  关键词是为了文献标引工作、用以表示全文主要内容信息的单词或术语。关键词不超过 5
  个,每个关键词中间用分号分隔。(模板作者注:关键词分隔符不用考虑,模板会自动处
  理。英文关键词同理。)
\end{cabstract}

\ckeywords{\TeX, \LaTeX, CJK, 模板, 论文}

\begin{eabstract}
   An abstract of a dissertation is a summary and extraction of research work
   and contributions. Included in an abstract should be description of research
   topic and research objective, brief introduction to methodology and research
   process, and summarization of conclusion and contributions of the
   research. An abstract should be characterized by independence and clarity and
   carry identical information with the dissertation. It should be such that the
   general idea and major contributions of the dissertation are conveyed without
   reading the dissertation.

   An abstract should be concise and to the point. It is a misunderstanding to
   make an abstract an outline of the dissertation and words ``the first
   chapter'', ``the second chapter'' and the like should be avoided in the
   abstract.

   Key words are terms used in a dissertation for indexing, reflecting core
   information of the dissertation. An abstract may contain a maximum of 5 key
   words, with semi-colons used in between to separate one another.
\end{eabstract}

\ekeywords{\TeX, \LaTeX, CJK, template, thesis}
% 中英文摘要
\tableofcontents% 目录
% 主要符号对照表
\begin{denotation}
\item[GNU] GNU's Not Unix /'gnu:/
\item[GFDL] GNU Free Documentation License
\item[GPL] GNU General Public License
\item[FSF] Free Software Foundation
\item[SMP] 对称多处理
\item[API] 应用程序编程接口
\item[$E$] 能量
\item[$m$] 质量
\item[$c$] 光速
\item[$P$] 概率
\item[$T$] 时间
\item[$v$] 速度
\end{denotation}
% 符号对照表
%%% 正文
\mainmatter%
\subfile{../body/chap01}
\subfile{../body/chap02}
\subfile{../body/chap03}
\subfile{../body/readme}
%%% 其它部分
\backmatter%
\begin{acknowledgement}% 致谢
衷心感谢导师 xxx 教授和物理系 xxx 副教授对本人的精心指导。他们的言传身教将使
我终生受益。

在美国麻省理工学院化学系进行九个月的合作研究期间,承蒙 xxx 教授热心指导与帮助,不
胜感激。感谢 xx 实验室主任 xx 教授,以及实验室全体老师和同学们的热情帮助和支
持!本课题承蒙国家自然科学基金资助,特此致谢。

感谢 \tongjithesis{},它的存在让我的论文写作轻松自在了许多,让我的论文格式规整漂亮了
许多。

虽然学习 \LaTeX 的过程并不轻松。

\end{acknowledgement}
% 致谢
%% 索引,星号影响是否出现在目录页中
\listoffigures%*% 插图索引
\listoftables%*% 表格索引
\listofcode% 代码索引
% \lisfoflistings% 代码索引
\listofequations*% 公式索引
\printTJbibliography% 参考文献
\begin{appendix}% 附录
  \subfile{../pages/appendix.tex}
\end{appendix}
\begin{resume} %简历、论文、成果

\resumesect{个人简历:}
\noindent xxxx 年 xx 月 xx 日出生于 xx 省 xx 县。\\
\noindent xxxx 年 9 月考入 xx 大学 xx 系 xx 专业,xxxx 年 7 月本科毕业并获得 xx 学士学位。\\
\noindent xxxx 年 9 月免试进入 xx 大学 xx 系攻读 xx 学位至今。

\resumesect{发表论文:} % 发表的和录用的合在一起
\begin{enumerate}[{[}1{]}]
\item Yang Y, Ren T L, Zhang L T, et al. Miniature microphone with silicon-
  based ferroelectric thin films. Integrated Ferroelectrics, 2003,
  52:229-235. (SCI 收录, 检索号:758FZ.)
\item 杨轶, 张宁欣, 任天令, 等. 硅基铁电微声学器件中薄膜残余应力的研究. 中国机
  械工程, 2005, 16(14):1289-1291. (EI 收录, 检索号:0534931 2907.)
\item 杨轶, 张宁欣, 任天令, 等. 集成铁电器件中的关键工艺研究. 仪器仪表学报,
  2003, 24(S4):192-193. (EI 源刊.)
\item Yang Y, Ren T L, Zhu Y P, et al. PMUTs for handwriting recognition. In
  press. (已被 Integrated Ferroelectrics 录用. SCI 源刊.)
\item Wu X M, Yang Y, Cai J, et al. Measurements of ferroelectric MEMS
  microphones. Integrated Ferroelectrics, 2005, 69:417-429. (SCI 收录, 检索号
  :896KM.)
\item 贾泽, 杨轶, 陈兢, 等. 用于压电和电容微麦克风的体硅腐蚀相关研究. 压电与声
  光, 2006, 28(1):117-119. (EI 收录, 检索号:06129773469.)
\item 伍晓明, 杨轶, 张宁欣, 等. 基于MEMS技术的集成铁电硅微麦克风. 中国集成电路,
  2003, 53:59-61.
\end{enumerate}

\resumesect{研究成果:} % 有就写,没有就删除
\begin{enumerate}[{[}1{]}]
\item 任天令, 杨轶, 朱一平, 等. 硅基铁电微声学传感器畴极化区域控制和电极连接的
  方法: 中国, CN1602118A. (中国专利公开号.)
\item Ren T L, Yang Y, Zhu Y P, et al. Piezoelectric micro acoustic sensor
  based on ferroelectric materials: USA, No.11/215, 102. (美国发明专利申请号.)
\end{enumerate}

\end{resume}
% 个人简历
%
\end{document}
