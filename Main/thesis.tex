\documentclass[degree=doctor,bibtype=numeric]{../THEME/tongjithesis}
% 选项列表(*代表默认选项)
%   degree=[master*|doctor], % 必选:硕士* 或 博士
%   bibtype=[numeric*|authoryear], % 可选:数字式上标* 或 作者-年份
%   degreetype=[academic*|profession|equaleducation], % 可选:学术型* 或 专业型 或 同等学力
%   titlenum=[rmtitlenum*|sftitlenum|heititlenum], % 可选:章节编号字体::有衬线* 或 无衬线 或 黑体
%   electronic=false, % 可选:封面左上角“打印时删除”显示与否
%   draft=false, % 可选:是否减少空白页以求页面紧凑,会在封面打印“电子版”字样作为提示。
%   secret=false, % 可选:是否保密
%   raggedbottom=true, % 可选:优化页面布置

% 所有其它可能用到的包都统一放到这里了,可以根据自己的实际添加或者删除。
\usepackage{../THEME/tongjithesis}

%参考文献更新使用biblatex包, 使用gb7714-2015标准, 具体参数设置可在cls文件中搜索biblatex进行了解
%加入bib文件(老版本文件依然能够使用)
\addbibresource{../ref/refs.bib}%

% 定义所有的eps文件在 figures 子目录下
\graphicspath{{../figures/}{../THEME}}

\begin{document}

\makecover*{../pages/cover} % 中英文封面,带*则无书脊
\makeauthorizationpage%[../pages/scanauth.pdf] % 授权书
\makedeclarepage%[../pages/scandecl.pdf] % 原创声明

\frontmatter %%% 论文前段
\input{../pages/abstract} % 中英文摘要
\tableofcontents % 目录
\input{../pages/denotation} % 符号对照表

%%% 正文
\mainmatter
\subfile{../body/chap01}
\subfile{../body/chap02}

\backmatter %%% 其它部分
\input{../pages/ack} % 致谢
%% 索引,星号影响是否出现在目录页中
\listoffigures%*% 插图索引
\listoftables%*% 表格索引
\lisfoflistings% 代码索引
\listofequations*% 公式索引
\printTJbibliography% 参考文献
\begin{appendix}% 附录
  \subfile{../pages/appendix.tex}
\end{appendix}
\input{../pages/resume} % 个人简历

\end{document}
